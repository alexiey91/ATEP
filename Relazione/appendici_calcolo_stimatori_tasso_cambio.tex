  \chapter{Calcolo Stimatori Tasso di Cambio}
	Ci siamo serviti degli strumenti sopra descritti (variabile aleatoria \textbf{t-Student} e \textbf{intervalli di confidenza}) per stimare con un certo \textbf{livello di confidenza} quale fosse l'intervallo di definizione del nostro parametro di interesse (il tasso di cambio atteso). Più precisamente abbiamo dovuto far ricorso ad una \textbf{quantità pivotale} di distribuzione \textbf{t-Student} poichè la varianza teorica del campione preso in considerazione è incognita e pertanto la nostra stima intervallare si basa sulla \textbf{varianza campionaria corretta}. L'intervallo di confidenza inserito nel seguito è stato utilizzato con un valore del \textbf{quantile} della \textbf{t-Student} pari alla radice della numerosità del campione in esame ($\sqrt{154}$), assicurando un \textbf{livello di confidenza} molto elevato (approssimabile a \textbf{1}). S\ped{n} sta ad identificare la \textbf{varianza campionaria corretta}, $\bar{X\ped{n}}$ la \textbf{media campionaria}, $t\ped{1-\frac{\alpha}{2},n-1}$ il quantile della \textbf{t-Student} con n gradi di libertà per il quale sfruttiamo la simmetria della distribuzione.

\[
\{\bar{X\ped{n}} - \frac{S\ped{n}}{\sqrt{n}}t\ped{1-\frac{\alpha}{2},n-1} \leq \mu \leq \bar{X\ped{n}} + \frac{S\ped{n}}{\sqrt{n}}t\ped{1-\frac{\alpha}{2},n-1}\}
\]  
  
    \label{sec:stimatori_tasso_cambio}