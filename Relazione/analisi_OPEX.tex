prova \ac{Fibank Albania}

Gli OPEX stimati mensilmente sono:

%
%	Tabella relativa ai OPEX sostenuti mensilemente
%
\begin{savenotes}
\begin{table}[htb]
\centering
 \caption{OPEX}
 \begin{tabular}{p{7cm}D{,}{,}{5.2}}
 \toprule
 	& \multicolumn{1}{c}{\textbf{Costo Mensile (\euro)}} \\
 \midrule
	& \\ 	
 	\makebox[7cm][l]{\textbf{Utenze}} & \\ 
	\makebox[7cm][r]{Telefono \footnote{comprende anche l'abbonamento ADSL}} & 36,00 \\
	\makebox[7cm][r]{Abbonamento Skype \footnote{Abbonamento World unlimited mins}} & 237,90 \\	
	\makebox[7cm][r]{Acqua} & 45,88 \\
	\makebox[7cm][r]{Luce} & 31,77 \\
	\makebox[7cm][r]{Gas} & 252,62 \\
	& \\
 	\makebox[7cm][l]{\textbf{Sorveglianza 24h}} & 19\thinspace 709,41 \\
	& \\ 	
 	\makebox[7cm][l]{\textbf{Pulizie}} & 8\thinspace 880,00 \\ 	
 	& \\
 	\makebox[7cm][l]{\textbf{Cancelleria}} & 500,00 \\ 
 	& \\
	\makebox[7cm][l]{\textbf{Stipendi Dipendenti}} & 25\thinspace 437,69 \\ 	 	
	& \\
	\makebox[7cm][l]{\textbf{Affitto Locale}} & 5\thinspace 000,00 \\
	& \\	
	\midrule
	\makebox[7cm][l]{\textbf{TOTALE (\euro)}} & 60\thinspace 131,27 \\		
 \bottomrule
 \end{tabular} 
 \label{table:opex}
\end{table}
\end{savenotes}

\subsection[Acqua]{Acqua}
Considerando un consumo medio giornaliero di $ 55 \: litri $ per persona, corrispondenti a:
	\begin{equation}
	\label{eq:consumo_acqua_giorno_dipendente}
	\begin{split}
		\frac{55}{1\thinspace 000} = 0,055 \: m^3
	\end{split}
	\end{equation}	
per la nostra azienda si stima, quindi un consumo giornaliero, per 34 persone, di:
	\begin{equation}
	\label{eq:consumo_acqua_giorno_azienda}
	\begin{split}
		\frac{55}{1\thinspace 000} \cdot 34 = 1,87 \: m^3
	\end{split}
	\end{equation}	
quindi, in un mese ( 22 giorni lavorativi effettivi ):	
	\begin{equation}
	\label{eq:consumo_acqua_giorno_azienda}
	\begin{split}
		\frac{55}{1\thinspace 000} \cdot 34 \cdot 22 = 41,14 \: m^3
	\end{split}
	\end{equation}	
In Albania, la bolletta dell'acqua prevede le seguenti voci:
\begin{savenotes}
\begin{table}[htb]
\centering
 \caption{Costo \si{m^3} acqua}
 \begin{tabular}{p{4cm}D{,}{,}{5.2}}
 \toprule
 	& \multicolumn{1}{c}{\textbf{Costo (LEK) per \si{m^3}}}\\
 \midrule
	\makebox[4cm][r]{Acqua Potabile} & 120\\
	\makebox[4cm][r]{Servizio Fognatura} & 30\\
 \midrule
	\makebox[4cm][r]{\textbf{TOTALE}} & 150\\	
 \bottomrule
 \end{tabular} 
\end{table}
\end{savenotes}
Si prevede, quindi, una bolletta mensile (LEK):
	\begin{equation}
	\label{eq:bolletta_acqua_azienda}
	\begin{split}
		41,14 \cdot 150 = 6\thinspace 171 \: ( \simeq 45,88 \: \mbox{\euro}) 
	\end{split}
	\end{equation}	
In sintesi:
\begin{savenotes}
\begin{table}[htb]
\centering
 \caption{Bolletta dell'Acqua}
 \begin{tabular}{p{4cm}D{,}{,}{5.2}D{,}{,}{5.2}D{,}{,}{5.2}}
 \toprule
 	\textbf{Base Stima Consumo}& \multicolumn{1}{c}{\textbf{Quantita' \si{m^3}}} & \multicolumn{1}{c}{\textbf{Costo (LEK)}} & \multicolumn{1}{c}{\textbf{Costo (\euro)}}\\
 \midrule
	\makebox[4cm][c]{giornaliero} & 1,87 & 280,50 & 2,05\\
	\makebox[4cm][c]{mensile} & 41,14 & 6\thinspace 171,00 & 45,88\\	
 \bottomrule
 \end{tabular} 
\end{table}
\end{savenotes}
\subsection[Luce]{Luce}
Considerando un consumo medio annuale di $ 39 \: kWh/m^2 $ per un ufficio,
prevediamo, avendo un ufficio di $ 137 \: m^2 $ un consumo medio \textbf{mensile} pari a:
	\begin{equation}
	\label{eq:consumo_luce_mensile}
	\begin{split}
		\frac{39 \cdot 137}{12} = 445,25 \: kWh 
	\end{split}
	\end{equation}	 
tenendo conto un costo di $ 9,5 \: (ALL / kWh) \: ( \simeq 0,07 \: (\mbox{\euro} / kWh) ) $\cite{ere_prices}, ci aspettiamo una bolletta mensile pari a:
	\begin{equation}
	\label{eq:bolletta_luce_mensile}
	\begin{split}
		445,25 \cdot 0,07 = 31,77 \: \mbox{\euro} 
	\end{split}
	\end{equation}	

\begin{savenotes}
\begin{table}[htb]
\centering
 \caption{Bolletta della Luce}
 \begin{tabular}{p{4cm}D{,}{,}{5.2}D{,}{,}{5.2}}
 \toprule
 	\textbf{Base Stima Consumo}& \multicolumn{1}{c}{\textbf{Quantita' (kWh)}} & \multicolumn{1}{c}{\textbf{Costo (\euro)}}\\
 \midrule
	\makebox[4cm][c]{annuale} & 5\thinspace 343,00 &  347,01\\
	\makebox[4cm][c]{mensile} & 445,25 & 31,77 \\	
 \bottomrule
 \end{tabular} 
\end{table}
\end{savenotes}
\subsection[Gas]{Gas}
Considerando un consumo medio annuale di $ 81 \cdot 10^6 \: m^3 $ e il costo di trasmissione a Tirana pari a $ 39 \: \$ $ per milione di metri cubi, possiamo determinare:
\begin{itemize}
\item la \textbf{bolletta annuale}
	\begin{equation}
	\label{eq:bolletta_gas_annuale}
	\begin{split}
		81 \cdot 39 = 3\thinspace 159 \: \$ 
	\end{split}
	\end{equation}	
\item la \textbf{bolletta mensile}
	\begin{equation}
	\label{eq:bolletta_gas_mensile}
	\begin{split}
		\frac{81 \cdot 39}{12} = 263,25 \: \$ \: ( \simeq 252,62 \: \mbox{\euro} )\footnotemark
	\end{split}
	\end{equation}		
\end{itemize}

\footnotetext{ cambio 1 \mbox{\euro} : 1,0425 \$}


\begin{savenotes}
\begin{table}[htb]
\centering
 \caption{Bolletta del Gas}
 \begin{tabular}{p{4cm}D{,}{,}{5.2}D{,}{,}{5.2}D{,}{,}{5.2}}
 \toprule
 	\textbf{Base Stima Consumo}& \multicolumn{1}{c}{\textbf{Quantita' (mln \si{m^3})}} & \multicolumn{1}{c}{\textbf{Costo (\$)}} & \multicolumn{1}{c}{\textbf{Costo (\euro)}}\\
 \midrule
	\makebox[4cm][c]{annuale} & 81,00 & 3\thinspace 159,00 & 3\thinspace 031,44\\
	\makebox[4cm][c]{mensile} & 6,75 & 263,25 & 252,62\\	
 \bottomrule
 \end{tabular} 
\end{table}
\end{savenotes}

	\begin{tcolorbox}[colframe=blue!75!black,adjusted title=\textbf{Osservazione!}]
		I calcoli effettuati in precedenza sono puramente teorici, in quanto si è stimato un consumo uniforme di acqua, luce e gas durante l'arco dell'anno. Ovviamente, ciò non corrisponde alla realtà in quanto nei mesi invernali si ha un consumo maggiore ed in quelli primaverili ed estivi uno minore. 
	\end{tcolorbox}