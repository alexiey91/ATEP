L'impatto dei giorni di malattia sull'analisi dei flussi mensili è stato valutato, considerando il numero medio di malati durante un anno solare.\newline A tal proposito è stato preso in esame un campione di 24 elementi corrispondenti al numero medio di giorni di malattia nell'\ac{UE} nel periodo dal 1990 al 2014.\newline
 \begin{tabular}{SS[table-format=2]}
 \toprule
 	{Anno} & {Giorni Malattia} \\
 \midrule
 	1990 & 10,7 \\
 	1991 & 11,1 \\
 	1992 & 11,1 \\
 	1993 & 11,4 \\
 	1994 & 11,4 \\
 	1995 & 11,6 \\
	1996 & 11,5 \\ 
	1997 & 11,3 \\
	1998 & 11,2 \\
	1999 & 11,5 \\
	2000 & 11,6 \\
	2001 & 11,8 \\
	2002 & 12,1 \\
	2003 & 12,2 \\
	2004 & 11,8 \\
	2006 & 11,4 \\
	2007 & 11,4 \\
	2008 & 11,6 \\
	2009 & 11,7 \\
	2010 & 11,6 \\
	2011 & 11,6 \\
	2012 & 11,7 \\							  
	2013 & 11,8 \\
	2014 & 11,8 \\

 \bottomrule
 \end{tabular} 
\newline
Le principali grandezze ad esso associato (i calcoli sono illustrati all'appendice \ref{sec:stimatori_malattie_dipendenti}sono pari a:

\begin{savenotes}
\begin{table}[htb]
\centering
 \caption{Statistiche}
 \begin{tabular}{p{5cm}D{,}{,}{5.5}}
 \toprule
 	\multicolumn{1}{c}{\textbf{Statistica}} & \multicolumn{1}{c}{\textbf{Valore}} \\
 \midrule 		
	\makebox[5cm][r]{Media Campionaria} & 12,03913\\
 	\makebox[5cm][r]{Varianza Campionaria Corretta} & 0,38525\\
 	\makebox[5cm][r]{Deviazione Standard Corretta} & 0,62068\\	
 \bottomrule
 \end{tabular} 
\end{table}
\end{savenotes} 

L'intervallo di confidenza, di livello $0,99$ ad esse associato è pari a:

\[	\left [ 11,4184 ; 12,6598 \right]		\] 
 
 
Per un singolo centralinista, consideriamo le seguenti ipotesi per un singolo mese: 
  
\begin{savenotes}
\begin{table}[htb]
\centering
 \caption{Assunzioni Centralinista}
 \begin{tabular}{p{7cm}D{,}{,}{5.2}}
 \toprule
 	& \multicolumn{1}{c}{\textbf{Quantita'}} \\
 \midrule 	
	\makebox[7cm][r]{Giorni lavorativi in un mese} & 18,50\\
	\makebox[7cm][r]{Giorni lavorativi in un anno} & 222,00\\
	\makebox[7cm][r]{Giorni assenza\footnote{pari alla media campionaria $ 12,03913 \simeq 12,00 $}} & 12,00\\	
 	\makebox[7cm][r]{Probabilita' di stipulare un contratto (\%)\footnote{dati istat}} & 15,60\\
 \bottomrule
 \end{tabular} 
\end{table}
\end{savenotes}
%
%	Numero chiamate centralinisti
%
\begin{savenotes}
\begin{table}[htb]
\centering
 \caption{Numero contratti 29 centralinisti}
 \begin{tabular}{p{5cm}D{,}{,}{5.2}}
 \toprule
 	& \multicolumn{1}{c}{\textbf{Quantità}} \\
 \midrule 		
	\makebox[5cm][r]{Numero chiamate annuali} & 605172,00\\
 	\makebox[5cm][r]{Numero contratti annuali} & 94406,83 \\
 	\makebox[5cm][r]{Numero contratti mensili} & 7867,24\\  	
 \bottomrule
 \end{tabular} 
\end{table}
\end{savenotes}

\begin{comment}

\begin{savenotes}
\begin{table}[htb]
\centering
 \caption{Numero contratti centralinisti}
 \begin{tabular}{p{7cm}D{,}{,}{5.2}D{,}{,}{5.2}}
 \toprule
 	& \multicolumn{1}{c}{\textbf{Singolo Centralinista}} & \multicolumn{1}{c}{\textbf{29 Centralinisti}} \\
 \midrule 		
	\makebox[7cm][r]{Numero chiamate giornaliere} & 94,00 & 2820\\
 	\makebox[7cm][r]{Numero chiamate mensili} & 1739,00\footnote{(94,00*18,50)} & 52170\\
 	\makebox[7cm][r]{Contratti stipulati mensilmente} & 271,29 & 8138,52\\ 	
 	\makebox[7cm][r]{Chiamate mensili in caso di assenza per 12 giorni} & 611,00\footnote{(94,00*(18,50-12,00))} & 18330\\
 	\makebox[7cm][r]{Contratti stipulati in caso di assenza per 12 giorni} & 95,32\footnote{(611*0,156)} & 2859,48\\ 	
 \bottomrule
 \end{tabular} 
\end{table}
\end{savenotes}

\end{comment}
 
\begin{savenotes}
\begin{table}[htb]
\centering
 \caption{Variazione Fatturato}
 \begin{tabular}{p{8cm}D{,}{,}{5.2}D{,}{,}{5.2}}
 \toprule
 	& \multicolumn{1}{c}{\textbf{VAN PAREGGIO}} & \multicolumn{1}{c}{\textbf{VAN CASO REALE}} \\
 \midrule
 	\makebox[8cm][r]{Probabilita' di successo di un contratto (\%)} & 11,54 & 15,00\\
 \midrule
	\makebox[8cm][r]{Numero contratti stipulati (1 mese)} & 8138,52 & 8138,52\\
	\makebox[8cm][r]{Numero contratti stipulati (30 malati in 1 mese)} & 2859,48 & 2859,48\\
 \midrule	
	\makebox[8cm][r]{Numero contratti successo (1 mese)} & 939,50 & 1220,78\\
	\makebox[8cm][r]{Numero contratti successo (30 malati in 1 mese)} & 329,98 & 428,92\\
 \midrule	
	\makebox[8cm][r]{Fatturato Lordo (\euro) (1 mese)} & 75160,00 & 97662,40\\
	\makebox[8cm][r]{Fatturato Lordo (\euro) (30 malati in 1 mese)} & 26398,72 & 34313,76\\	
 \midrule	
	\makebox[8cm][r]{Fatturato Netto (\euro) (1 mese)} & 65314,04 & 84917,46\\
	\makebox[8cm][r]{Fatturato Netto (\euro) (30 malati in 1 mese)} & 22953,69 & 29835,81\\	
 \bottomrule
 \end{tabular} 
\end{table}\
\end{savenotes}

\section[Caso di Studio]{Caso di Studio}

%
%	Tabella relativa al caso teorico in cui non ci siano malati durante un anno
%
\begin{savenotes}
\begin{table}[htb]
\centering
 \caption{Variazione VAN (Caso Teorico)}
 \begin{tabular}{p{4cm}D{,}{,}{5.2}D{,}{,}{5.2}D{,}{,}{5.2}D{,}{,}{7.4}}
 \toprule
 	& \multicolumn{1}{c}{Flusso di cassa mensile (\euro)} & \multicolumn{1}{c}{Contratti Mensili } &\multicolumn{1}{c}{\textbf{VAN}}&\multicolumn{1}{c}{\textbf{ \% Contratti}} \\
 \midrule	
 	\makebox[4cm][r]{Pareggio} & 65\thinspace 355,07 & 939,18 & 0,00 & 0,1154\\ 
	\makebox[4cm][r]{Ottimo} & 553\thinspace 419,36 & 8\thinspace 138,52 & 5\thinspace 248\thinspace 057,65 & 1,0000\\
	\makebox[4cm][r]{Caso di Studio 20,0 \%} & 110\thinspace 683,87 & 1\thinspace 627,70 & 487\thinspace 411,50 & 0,2000\\
	\makebox[4cm][r]{Caso di Studio 15,0 \%} & 83\thinspace 012,90 & 1\thinspace 220,78 & 189\thinspace 871,11 & 0,1500 \\
	\makebox[4cm][r]{Caso di Studio 12,5 \%} & 69\thinspace 177,42 & 1\thinspace 017,32 & 41\thinspace 100,92 & 0,1250 \\
 \bottomrule
 \end{tabular} 
\end{table}
\end{savenotes}


\subsection[Caso Uniforme]{Caso Uniforme}

%
%	Tabella relativa al caso in cui il numero di malati si distribuisce in maniera uniforme durante l'anno
%
\begin{savenotes}
\begin{table}[htb]
\centering
 \caption{Variazione VAN (Malati distribuiti uniformemente)}
 \begin{tabular}{p{4cm}D{,}{,}{5.2}D{,}{,}{5.2}D{,}{,}{5.2}D{,}{,}{7.4}}
 \toprule
 	& \multicolumn{1}{c}{Flusso di cassa mensile (\euro)} & \multicolumn{1}{c}{Contratti Mensili } &\multicolumn{1}{c}{\textbf{VAN}}&\multicolumn{1}{c}{\textbf{ \% Contratti}} \\
 \midrule	 
	\makebox[4cm][r]{Ottimo} & 615\thinspace 888,00 & 7\thinspace 698,60 & 4\thinspace 926\thinspace 392,37 & 1,0000\\
	\makebox[4cm][r]{Caso di Studio 15,0 \%} & 92\thinspace 406,24 & 1\thinspace 155,078 & 141\thinspace 831,90 & 0,1500 \\
 \bottomrule
 \end{tabular} 
\end{table}
\end{savenotes}

\subsection[Caso Ottimo]{Caso Ottimo} 

%
%	Tabella relativa al caso in cui il numero di malati si concentra nel mese di Dicembre
%
\begin{savenotes}
\begin{table}[htb]
\centering
 \caption{Variazione VAN (Malati nel mese di Dicembre)}
 \begin{tabular}{p{4cm}D{,}{,}{5.2}D{,}{,}{5.2}D{,}{,}{5.2}}
 \toprule
 	& \multicolumn{1}{c}{Flusso di cassa mensile (\euro)} & \multicolumn{1}{c}{Contratti Mensili } &\multicolumn{1}{c}{\textbf{VAN}} \\
 \midrule	 
	\makebox[4cm][r]{Caso di Studio 15,0 \%} & 39\thinspace 592,80 & 494,91 & 150\thinspace 010,00 \\
 \bottomrule
 \end{tabular} 
\end{table}
\end{savenotes}

\subsection[Caso Peggiore]{Caso Peggiore}

%
%	Tabella relativa al caso in cui il numero di malati si concentra nel mese di Gennaio
%
\begin{savenotes}
\begin{table}[htb]
\centering
 \caption{Variazione VAN (Malati nel mese di Gennaio)}
 \begin{tabular}{p{4cm}D{,}{,}{5.2}D{,}{,}{5.2}D{,}{,}{5.2}}
 \toprule
 	& \multicolumn{1}{c}{Flusso di cassa mensile (\euro)} & \multicolumn{1}{c}{Contratti Mensili } &\multicolumn{1}{c}{\textbf{VAN}} \\
 \midrule	 
	\makebox[4cm][r]{Caso di Studio 15,0 \%} & 29\thinspace 034,40 & 362,93 & 132\thinspace 972,00 \\
 \bottomrule
 \end{tabular} 
\end{table}
\end{savenotes}
