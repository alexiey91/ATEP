 \begin{tabular}{SS[table-format=2]}
 \toprule
 	{Anno} & {Giorni Malattia} \\
 \midrule
 	1990 & 10,7 \\
 	1991 & 11,1 \\
 	1992 & 11,1 \\
 	1993 & 11,4 \\
 	1994 & 11,4 \\
 	1995 & 11,6 \\
	1996 & 11,5 \\ 
	1997 & 11,3 \\
	1998 & 11,2 \\
	1999 & 11,5 \\
	2000 & 11,6 \\
	2001 & 11,8 \\
	2002 & 12,1 \\
	2003 & 12,2 \\
	2004 & 11,8 \\
	2006 & 11,4 \\
	2007 & 11,4 \\
	2008 & 11,6 \\
	2009 & 11,7 \\
	2010 & 11,6 \\
	2011 & 11,6 \\
	2012 & 11,7 \\							  
	2013 & 11,8 \\
	2014 & 11,8 \\

 \bottomrule
 \end{tabular} 

Prendendo in esame il campione caratterizzato dai valori precedentemente esposti, si possono calcolare le seguenti statistiche di interesse:

\begin{savenotes}
\begin{table}[htb]
\centering
 \caption{Statistiche}
 \begin{tabular}{p{5cm}D{,}{,}{5.5}}
 \toprule
 	\multicolumn{1}{c}{\textbf{Statistica}} & \multicolumn{1}{c}{\textbf{Valore}} \\
 \midrule 		
	\makebox[5cm][r]{Media Campionaria} & 12,03913\\
 	\makebox[5cm][r]{Varianza Campionaria Corretta} & 0,38525\\
 	\makebox[5cm][r]{Deviazione Standard Corretta} & 0,62068\\	
 \bottomrule
 \end{tabular} 
\end{table}
\end{savenotes} 
 
Da questi valori si può determinare il seguente intervallo di confidenza con il 95 \% di attendibilità

\[	\left [ 11,4184 ; 12,6598 \right]		\] 
 
 
consideriamo le seguenti ipotesi per ogni singolo centralinista: 
\begin{savenotes}
\begin{table}[htb]
\centering
 \caption{Assunzioni iniziali in un singolo mese}
 \begin{tabular}{p{7cm}D{,}{,}{5.2}}
 \toprule
 	& \multicolumn{1}{c}{\textbf{Quantita'}} \\
 \midrule 	
	\makebox[7cm][r]{Giorni lavorativi in un mese} & 18,50\\
	\makebox[7cm][r]{Giorni lavorativi in un anno} & 222,00\\
	\makebox[7cm][r]{Giorni assenza} & 12,00\\	
 	\makebox[7cm][r]{Probabilita' di stipulare un contratto (\%)\footnote{dati istat}} & 15,60\\
 \bottomrule
 \end{tabular} 
\end{table}
\end{savenotes}
%
%	Numero chiamate centralinisti
%
\begin{savenotes}
\begin{table}[htb]
\centering
 \caption{Numero contratti 29 centralinisti}
 \begin{tabular}{p{5cm}D{,}{,}{5.2}}
 \toprule
 	& \multicolumn{1}{c}{\textbf{Quantità}} \\
 \midrule 		
	\makebox[5cm][r]{Numero chiamate annuali} & 605172,00\\
 	\makebox[5cm][r]{Numero contratti annuali} & 94406,83 \\
 	\makebox[5cm][r]{Numero contratti mensili} & 7867,24\\  	
 \bottomrule
 \end{tabular} 
\end{table}
\end{savenotes}

\begin{comment}

\begin{savenotes}
\begin{table}[htb]
\centering
 \caption{Numero contratti centralinisti}
 \begin{tabular}{p{7cm}D{,}{,}{5.2}D{,}{,}{5.2}}
 \toprule
 	& \multicolumn{1}{c}{\textbf{Singolo Centralinista}} & \multicolumn{1}{c}{\textbf{29 Centralinisti}} \\
 \midrule 		
	\makebox[7cm][r]{Numero chiamate giornaliere} & 94,00 & 2820\\
 	\makebox[7cm][r]{Numero chiamate mensili} & 1739,00\footnote{(94,00*18,50)} & 52170\\
 	\makebox[7cm][r]{Contratti stipulati mensilmente} & 271,29 & 8138,52\\ 	
 	\makebox[7cm][r]{Chiamate mensili in caso di assenza per 12 giorni} & 611,00\footnote{(94,00*(18,50-12,00))} & 18330\\
 	\makebox[7cm][r]{Contratti stipulati in caso di assenza per 12 giorni} & 95,32\footnote{(611*0,156)} & 2859,48\\ 	
 \bottomrule
 \end{tabular} 
\end{table}
\end{savenotes}

\end{comment}
 
\begin{savenotes}
\begin{table}[htb]
\centering
 \caption{Variazione Fatturato}
 \begin{tabular}{p{8cm}D{,}{,}{5.2}D{,}{,}{5.2}}
 \toprule
 	& \multicolumn{1}{c}{\textbf{VAN PAREGGIO}} & \multicolumn{1}{c}{\textbf{VAN CASO REALE}} \\
 \midrule
 	\makebox[8cm][r]{Probabilita' di successo di un contratto (\%)} & 11,54 & 15,00\\
 \midrule
	\makebox[8cm][r]{Numero contratti stipulati (1 mese)} & 8138,52 & 8138,52\\
	\makebox[8cm][r]{Numero contratti stipulati (30 malati in 1 mese)} & 2859,48 & 2859,48\\
 \midrule	
	\makebox[8cm][r]{Numero contratti successo (1 mese)} & 939,50 & 1220,78\\
	\makebox[8cm][r]{Numero contratti successo (30 malati in 1 mese)} & 329,98 & 428,92\\
 \midrule	
	\makebox[8cm][r]{Fatturato Lordo (\euro) (1 mese)} & 75160,00 & 97662,40\\
	\makebox[8cm][r]{Fatturato Lordo (\euro) (30 malati in 1 mese)} & 26398,72 & 34313,76\\	
 \midrule	
	\makebox[8cm][r]{Fatturato Netto (\euro) (1 mese)} & 65314,04 & 84917,46\\
	\makebox[8cm][r]{Fatturato Netto (\euro) (30 malati in 1 mese)} & 22953,69 & 29835,81\\	
 \bottomrule
 \end{tabular} 
\end{table}\
\end{savenotes}