L'impatto dei giorni di malattia sull'analisi dei flussi mensili è stato valutato, considerando il numero medio di malati durante un anno solare.\newline A tal proposito è stato preso in esame un campione di 23 elementi corrispondenti al numero medio di giorni di malattia nell'\ac{UE} nel periodo dal 1990 al 2014.\newline
%
%	Campione del numero di malati nell'ambito dell'Unione Europea
%
\begin{savenotes}
\begin{table}[htb]
\centering
 \caption{Distribuzione Numero Giorni Malattia \ac{UE}}
 \begin{tabular}{D{,}{,}{5.1}D{,}{,}{5.1}}
 \toprule
 	\multicolumn{1}{c}{\textbf{Anno}} & \multicolumn{1}{c}{\textbf{Giorni Malattia (in media)}} \\
 \midrule 		
 	1990 & 10,7 \\
 	1991 & 11,1 \\
 	1992 & 11,1 \\
 	1993 & 11,4 \\
 	1994 & 11,4 \\
 	1995 & 11,6 \\
	1996 & 11,5 \\ 
	1997 & 11,3 \\
	1998 & 11,2 \\
	1999 & 11,5 \\
	2000 & 11,6 \\
	2001 & 11,8 \\
	2002 & 12,1 \\
	2003 & 12,2 \\
	2004 & 11,8 \\
	2006 & 11,4 \\
	2007 & 11,4 \\
	2008 & 11,6 \\
	2009 & 11,7 \\
	2010 & 11,6 \\
	2011 & 11,6 \\
	2012 & 11,7 \\							  
	2013 & 11,8 \\
	2014 & 11,8 \\
 \bottomrule
 \end{tabular} 
\end{table}
\end{savenotes}

Le principali grandezze ad esso associato (per approfondimenti si rimanda all'appendice \ref{sec:stimatori_malattie_dipendenti}) sono pari a:

\begin{savenotes}
\begin{table}[htb]
\centering
 \caption{Grandezze}
 \begin{tabular}{p{6cm}D{,}{,}{5.5}}
 \toprule
 	\multicolumn{1}{c}{\textbf{Grandezza}} & \multicolumn{1}{c}{\textbf{Valore}} \\
 \midrule 		
	\makebox[6cm][r]{Media Campionaria} & 12,03913\\
 	\makebox[6cm][r]{Varianza Campionaria Corretta} & 0,38525\\
 	\makebox[6cm][r]{Deviazione Standard Corretta} & 0,62068\\	
 \bottomrule
 \end{tabular} 
 \label{table:numero_medio_malattia}
\end{table}
\end{savenotes} 

L'intervallo di confidenza, di livello $0,9$ ad esse associato è pari a:
\begin{equation}
	\label{eq:intervallo_confidenza_num_malattia}
	\begin{split}
		\left [ 11,4184 ; 12,6598 \right]	 
	\end{split}
\end{equation}
\subsection[Casi di Studio]{Casi di Studio}
Si sono analizzati, nel dettaglio l'impatto dei dipendenti che si assentano per malattia:
\begin{itemize}
\item \textbf{uniformemente} durante l'arco dell'anno (\textbf{CASO MEDIO});
\item \textbf{concentrati} nel mese di Dicembre (\textbf{CASO MIGLIORE});
\item \textbf{concentrati} nel mese di Gennaio (\textbf{CASO PEGGIORE});
\end{itemize} 
%
%	Tabella relativa al caso teorico in cui non ci siano malati durante un anno
%
\begin{comment}
\begin{savenotes}
\begin{table}[htb]
\centering
 \caption{Variazione VAN (Caso Teorico)}
 \begin{tabular}{p{4cm}D{,}{,}{5.2}D{,}{,}{5.2}D{,}{,}{5.2}D{,}{,}{7.4}}
 \toprule
 	& \multicolumn{1}{c}{Flusso di cassa mensile (\euro)} & \multicolumn{1}{c}{Contratti Mensili } &\multicolumn{1}{c}{\textbf{VAN}}&\multicolumn{1}{c}{\textbf{ \% Contratti}} \\
 \midrule	
 	\makebox[4cm][r]{Pareggio} & 65\thinspace 355,07 & 939,18 & 0,00 & 0,1154\\ 
	\makebox[4cm][r]{Ottimo} & 553\thinspace 419,36 & 8\thinspace 138,52 & 5\thinspace 248\thinspace 057,65 & 1,0000\\
	\makebox[4cm][r]{Caso di Studio 20,0 \%} & 110\thinspace 683,87 & 1\thinspace 627,70 & 487\thinspace 411,50 & 0,2000\\
	\makebox[4cm][r]{Caso di Studio 15,0 \%} & 83\thinspace 012,90 & 1\thinspace 220,78 & 189\thinspace 871,11 & 0,1500 \\
	\makebox[4cm][r]{Caso di Studio 12,5 \%} & 69\thinspace 177,42 & 1\thinspace 017,32 & 41\thinspace 100,92 & 0,1250 \\
 \bottomrule
 \end{tabular} 
\end{table}
\end{savenotes}
\end{comment}

\subsubsection[Caso Medio]{Caso Medio}
Si suppone che i giorni di malattia (\ref{table:numero_medio_malattia}) annuali per singolo dipendente siano distribuiti in maniera uniforme durante l'arco dell'anno, quindi, è (ripetendo questo ragionamento per tutti i 30 centralinisti) come se un centralinista si assentasse per un giorno al mese. \newline
La squadra di centralinisti pienamente operativa è quindi costituita da 29 persone. \newline
Tenendo conto che un singolo centralinista è in grado potenzialmente di contattare un numero di clienti interessati (in un mese di 30 giorni) pari a:
	\begin{equation}
	\label{eq:chiamate_1_dipendente}
	\begin{split}
		30 \cdot 94 \cdot 0,156 = 439,92
	\end{split}
	\end{equation}
ogni mese, il numero di potenziali clienti contattati teoricamente sarà pari a:
	\begin{equation}
	\label{eq:chiamate_29_dipendenti}
	\begin{split}
		\underbrace{8\thinspace 138,52}_{chiamate\: 30 \: centralinisti} - 439,92 = 7\thinspace 698,60
 	\end{split}
	\end{equation}
quindi, il fatturato mensile al netto dell'IVA sarà pari a:
	\begin{equation}
	\label{eq:fatturato_mensile_29_dipendenti}
	\begin{split}
		7\thinspace 698,60 \cdot 80,00 \:\mbox{\euro} \: \cdot 0,8 = 492\thinspace 710,40 \: \:\mbox{\euro}
 	\end{split}
	\end{equation}	
dato (\ref{eq:fatturato_mensile_29_dipendenti}) il valore del \textbf{\ac{VAN}}(\ref{eq:van_caso_studio_2}) corrispodente sarà pari a:
	\begin{eqnarray}
	\label{eq:van_fatturato_mensile_29_dipendenti}
		y(492\thinspace 710,40) & = & -565\thinspace 875,32 + 8,47654 \cdot 492\thinspace 710,40 \nonumber \\
								 & = & 3\thinspace 610\thinspace 604,09 
	\end{eqnarray}	
Un discorso analogo lo si può attuare nel caso di studio del 15 \%. \newline
In sintesi, si registrano i seguenti valori:
%
%	Tabella relativa al caso in cui il numero di malati si distribuisce in maniera uniforme durante l'anno
%
\begin{savenotes}
\begin{table}[htb]
\centering
 \caption{Variazione VAN (Malati distribuiti uniformemente)}
 \label{table:van_malati_uniforme_anno}
 \begin{tabular}{p{4cm}D{,}{,}{5.2}D{,}{,}{5.2}D{,}{,}{5.2}D{,}{,}{7.4}}
 \toprule
 	& \multicolumn{1}{c}{Flusso di cassa mensile (\euro)} & \multicolumn{1}{c}{Contratti Mensili } &\multicolumn{1}{c}{\textbf{VAN (\euro)}}&\multicolumn{1}{c}{\textbf{ \% Contratti}} \\
 \midrule	 
	\makebox[4cm][r]{Ottimo} & 492\thinspace 710,40 & 7\thinspace 698,60 & 3\thinspace 610\thinspace 604,09 & 1,0000\\
	\makebox[4cm][r]{Caso di Studio 15,0 \%} & 73\thinspace 925,12 & 1\thinspace 155,08 & 60\thinspace 753,92 & 0,1500 \\
 \bottomrule
 \end{tabular} 
\end{table}
\end{savenotes}

\subsubsection[Caso Migliore - Mese Dicembre]{Caso Migliore - Mese Dicembre} 
La situazione migliore, si registra quando tutti i centralinisti si assentano per 11 giorni (pari al limite inferiore dell'intervallo di confidenza (\ref{eq:intervallo_confidenza_num_malattia}) arrotondato per difetto) nel mese di Dicembre. \newline
Il numero totale di chiamate \underline{non effettuate}, questo mese, sarà pari quindi a: 
	\begin{equation}
	\label{eq:totale_chiamate_non_effettuate_dicembre}
	\begin{split}
		\underbrace{11}_{giorni \: malattia} \cdot \underbrace{30}_{num\: dipendenti} \cdot \underbrace{94}_{num \: medio \: chiamate} \cdot 0,156 \cdot 0,15 = 725,87 
 	\end{split}
	\end{equation}
quelle realizzate sarà uguale a:
	\begin{equation}
	\label{eq:totale_chiamate_effettuate_dicembre}
	\begin{split}
		\underbrace{1\thinspace 220,78}_{numero\: chiamate \: 30 \: dipendenti} - 725,87 = 494,98
 	\end{split}
	\end{equation}
ed, infine, il fatturato, al netto dell'IVA:
	\begin{equation}
	\label{eq:fatturato_dicembre}
	\begin{split}
		494,98 \cdot 80,00 \cdot 0,8 = 31\thinspace 678,72 \: \mbox{\euro}
 	\end{split}
	\end{equation}
Il calcolo del VAN ad esso associato non è pari all'applicazione della formula (\ref{eq:van_caso_studio_2}), ma la formula da utilizzare è la seguente:
	\begin{equation}
	\label{eq:van_caso_dicembre}
	\begin{split}
		y(x) = -56\thinspace 170,20 + \frac{31\thinspace 678,72 - 60\thinspace 131,27}{(1+0,058)^{12}} + \sum_{k=1}^{11} \frac{x - 60\thinspace 131,27} {(1+0,058)^{k}}
 	\end{split}
	\end{equation}
dove la quantità:
	\begin{equation}
	\label{eq:attualizzazione_dicembre}
	\begin{split}
		\frac{31\thinspace 678,72 - 60\thinspace 131,27}{(1+0,058)^{12}} = -14\thinspace 464,1589 \simeq -14\thinspace 464,16
 	\end{split}
	\end{equation}
è pari all'attualizzazione dell'utile del mese di Dicembre al 1 Gennaio 2016, 
\newline mentre la quantità: 
	\begin{eqnarray}
	\label{eq:attualizzazione_utile_mesi}
		\sum_{k=1}^{11} \frac{x - 60\thinspace 131,27} {(1+0,058)^{k}} & = & \nonumber \\
		& = & (x - 60\thinspace 131,27) \cdot \underbrace{\sum_{k=1}^{11}\frac{1}{(1+0,058)^{k}}}_{{}=7,9682} \nonumber \\
		& = & (x - 60\thinspace 131,27) \cdot 7,9682 
	\end{eqnarray}
rappresenta l'attualizzazione al 1 Gennaio 2016 dell'utile realizzato negli 11 mesi precedenti (Gennaio \ldots Novembre).\newline
Combinando (\ref{eq:attualizzazione_dicembre}) e (\ref{eq:attualizzazione_utile_mesi}), la (\ref{eq:van_caso_dicembre}) diventa:
	\begin{eqnarray}
	\label{eq:van_caso_dicembre_finale}
		y(x) & = & -56\thinspace 170,20 - 14\thinspace 464,16 + (x - 60\thinspace 131,27) \cdot 7,9682 \nonumber \\
			 & = & -70\thinspace 634,36 + (x - 60\thinspace 131,27) \cdot 7,9682  \nonumber \\
			 & = & -549\thinspace 772,3456 + x \cdot 7,9682 
	\end{eqnarray}
\newline

la (\ref{eq:van_caso_dicembre_finale}) calcolata per $ x = 78\thinspace 129,81 \: \mbox{\euro}$ (tabella \ref{table:stime_guadagni}) è pari a:
	\begin{eqnarray}
	\label{eq:van_dicembre}
		y(78\thinspace 129,81) & = & -549\thinspace 772,3456 + 78\thinspace 129,81 \cdot 7,9682 \nonumber \\
								& = & 7\thinspace 278,6064 \: \mbox{\euro} \simeq 72\thinspace 781,61 \: \mbox{\euro}  
	\end{eqnarray}
 	

%
%	Tabella relativa al caso in cui il numero di malati si concentra nel mese di Dicembre
%
\begin{savenotes}
\begin{table}[htb]
\centering
 \caption{Variazione VAN (Malati nel mese di Dicembre)}
 \begin{tabular}{p{4cm}D{,}{,}{5.2}D{,}{,}{5.2}D{,}{,}{5.2}}
 \toprule
 	& \multicolumn{1}{c}{Flusso di cassa mensile (\euro)} & \multicolumn{1}{c}{Contratti Mensili} &\multicolumn{1}{c}{\textbf{VAN (\euro)}} \\
 \midrule	 
	\makebox[4cm][r]{Caso di Studio 15,0 \%} & 31\thinspace 678,72 & 494,98 & 72\thinspace 781,61 \\
 \bottomrule
 \end{tabular} 
\end{table}
\end{savenotes}

\subsubsection[Caso Peggiore - Mese di Gennaio]{Caso Peggiore - Mese di Gennaio}
La situazione peggiore, si registra, invece, quando tutti i centralinisti si assentano per 13 giorni (pari al limite superiore dell'intervallo di confidenza (\ref{eq:intervallo_confidenza_num_malattia}) arrotondato per eccesso) nel mese di Gennaio.
\newline
Il numero totale di chiamate \underline{non effettuate}, questo mese, sarà pari quindi a: 
	\begin{equation}
	\label{eq:totale_chiamate_non_effettuate_gennaio}
	\begin{split}
		\underbrace{13}_{giorni \: malattia} \cdot \underbrace{30}_{num\: dipendenti} \cdot \underbrace{94}_{num \: medio \: chiamate} \cdot 0,156 \cdot 0,15 = 857,844 
 	\end{split}
	\end{equation}
quelle realizzate sarà uguale a:
	\begin{equation}
	\label{eq:totale_chiamate_effettuate_gennaio}
	\begin{split}
		\underbrace{1\thinspace 220,78}_{numero\: chiamate \: 30 \: dipendenti} - \: 857,84 = 362,94
 	\end{split}
	\end{equation}	
ed, infine, il fatturato, al netto dell'IVA:
	\begin{equation}
	\label{eq:fatturato_gennaio}
	\begin{split}
		362,94 \cdot 80,00 \cdot 0,8 = 23\thinspace 228,16 \: \mbox{\euro}
 	\end{split}
	\end{equation}
Il calcolo del VAN ad esso associato non è pari all'applicazione della formula (\ref{eq:van_caso_studio_2}), ma la formula da utilizzare è la seguente:
	\begin{equation}
	\label{eq:van_caso_gennaio}
	\begin{split}
		y(x) = -56\thinspace 170,20 + \frac{23\thinspace 228,16 - 60\thinspace 131,27}{1+0,058} + \sum_{k=2}^{12} \frac{x - 60\thinspace 131,27} {(1+0,058)^{k}}
 	\end{split}
	\end{equation}
dove il valore:
	\begin{equation}
	\label{eq:attualizzazione_gennaio}
	\begin{split}
		\frac{23\thinspace 228,16 - 60\thinspace 131,27}{1+0,058} = -34\thinspace 880,0662 \simeq -34\thinspace 880,07
 	\end{split}
	\end{equation}
è pari all'attualizzazione dell'utile del 31 Gennaio al 1 Gennaio 2016, 
\newline mentre la quantità: 
	\begin{eqnarray}
	\label{eq:attualizzazione_utile_mesi_succ}
		\sum_{k=2}^{12} \frac{x - 60\thinspace 131,27} {(1+0,058)^{k}} & = & \nonumber \\
		& = & (x - 60\thinspace 131,27) \cdot \underbrace{\sum_{k=2}^{12}\frac{1}{(1+0,058)^{k}}}_{{}=7,5314} \nonumber \\
		& = & (x - 60\thinspace 131,27) \cdot 7,5314 
	\end{eqnarray}
rappresenta l'attualizzazione al 1 Gennaio 2016 dell'utile realizzato negli 11 mesi successivi (Febbraio \ldots Dicembre).\newline
Combinando (\ref{eq:attualizzazione_gennaio}) e (\ref{eq:attualizzazione_utile_mesi_succ}), la (\ref{eq:van_caso_gennaio}) diventa:
	\begin{eqnarray}
	\label{eq:van_caso_gennaio_finale}
		y(x) & = & -56\thinspace 170,20 - 34\thinspace 880,07 + (x - 60\thinspace 131,27) \cdot 7,5314 \nonumber \\
			 & = & -91\thinspace 050,27 + (x - 60\thinspace 131,27) \cdot 7,5314  \nonumber \\
			 & = & -543\thinspace 922,9169 + x \cdot 7,5314 
	\end{eqnarray}
\newline
la (\ref{eq:van_caso_gennaio_finale}) calcolata per $ x = 78\thinspace 129,81 \: \mbox{\euro}$ (tabella \ref{table:stime_guadagni}) è pari a:
	\begin{eqnarray}
	\label{eq:van_dicembre}
		y(78\thinspace 129,81) & = & -543\thinspace 922,9169 + 78\thinspace 129,81 \cdot 7,5314 \nonumber \\
								& = & 44\thinspace 503,9341 \: \mbox{\euro} \simeq 44\thinspace 503,93 \: \mbox{\euro}  
	\end{eqnarray}
%
%	Tabella relativa al caso in cui il numero di malati si concentra nel mese di Gennaio
%
\begin{savenotes}
\begin{table}[htb]
\centering
 \caption{Variazione VAN (Malati nel mese di Gennaio)}
 \begin{tabular}{p{4cm}D{,}{,}{5.2}D{,}{,}{5.2}D{,}{,}{5.2}}
 \toprule
 	& \multicolumn{1}{c}{Flusso di cassa mensile (\euro)} & \multicolumn{1}{c}{Contratti Mensili} &\multicolumn{1}{c}{\textbf{VAN (\euro)}} \\
 \midrule	 
	\makebox[4cm][r]{Caso di Studio 15,0 \%} & 23\thinspace 228,16 & 362,94 & 44\thinspace 503,93 \\
 \bottomrule
 \end{tabular} 
\end{table}
\end{savenotes}
