\section*{Elenco Acronimi}
\begin{acronym}
\begin{multicols}{2}
\acro{ALL}{\textit{\textbf{AL}banian \textbf{L}ek }}
		   {\newline \small La moneta in uso in Albania. E' definita dallo standard \textbf{ISO 4217}  \par}

\acro{ANV S.r.l.}{\textit{\textbf{A}gostini \textbf{N}anni \textbf{V}alenti \textbf{S.r.l.}}}
		   {\newline \small Società albanese operante nel settore dei call center facente parte del gruppo Sistelia \par}

\acro{CAPEX}{\textit{\textbf{C}apital \textbf{EX}penditure}}
		   {\newline \small Noto anche come \textit{spese per capitale} indicano i fondi utilizzati dalle imprese per acquistare asset durevoli  \par}

\acro{CEO}{\textit{\textbf{C}hief \textbf{E}xecutive \textbf{O}fficer }}
		   {\newline \small E' la figura del consiglio di amministrazione posto a capo del management aziendale. E' l'equivalente dell'amministratore delegato  \par}

\acro{Fibank Albania}{\textit{\textbf{F}irst \textbf{I}nvestment \textbf{B}ank Albania}}
		   {\newline \small Banca di investimento sussidiaria della Finbank Bulgaria, operante in Albania dal 27/06/2007\cite{finbank_al} \par}


\acro{IVA}{\textit{\textbf{I}mposta sul \textbf{V}alore \textbf{A}ggiunto }}
		   {\newline \small E' un imposta applicata sul valore aggiunto di ogni fase della produzione, di scambio di beni e servizi. E' indicata anche come \textbf{VAT} (\textit{\textbf{V}alue \textbf{A}dded \textbf{T}ax}), mentre in Albania è nota come \textbf{TVSH} (\textit{\textbf{T}atimi mbi \textbf{V}ler\"en e \textbf{SH}turar}) \par}

\acro{OPEX}{\textit{\textbf{OP}erating \textbf{EX}penditure}}
		   {\newline \small Noto anche come \textit{spesa operativa} è il costo per gestire un prodotto, un business o un sistema \par}

\acro{Sh.p.k}{\textit{\textbf{Sh}oq\"eri me \textbf{p}\"ergjegj\"esi t\"e \textbf{k}ufizuar}}
		   {\newline \small l'equivalente albanese dell'italiana \textbf{\ac{S.r.l.}}\par}

\acro{S.r.l.}{\textit{\textbf{S}ocietà a \textbf{r}esponsabilità \textbf{l}imitata}}
		   {\newline \small società di capitali, dotata di personalità giuridica. Risponde delle obbligazioni sociali nei limiti delle quote versate dai soci\par}

\acro{TIR}{\textit{\textbf{T}asso \textbf{I}nterno \textbf{R}endimento}}
		   {\newline \small noto anche come \textbf{IRR} (\textit{\textbf{I}nternal \textbf{R}ate of \textbf{R}eturn}) è pari al valore del tasso di attualizzazione \textit{i} tale da annullare il \ac{VAN} \par}		   

\acro{UE}{\textit{\textbf{U}nione \textbf{E}uropea}}
		   {\newline \small Un'organizzazione internazionale politica ed economica di carattere sovranazionale, comprendente di 28 paesi membri indipendenti e democratici dell'Europa\cite{paesiUE} \par}
		   
\acro{VAN}{\textit{\textbf{V}alore \textbf{A}ttuale \textbf{N}etto}}
		   {\newline \small il valore attuale di una serie di flussi di cassa che si realizzano in tempi futuri, attualizzati con il tasso di rendimento. \'E noto anche come \textbf{NPV} (\textit{\textbf{N}et \textbf{P}resent \textbf{V}alue}) \par}		   

\acro{WACC}{\textit{\textbf{W}eighted \textbf{A}verage \textbf{C}ost of \textbf{C}apital}}
		   {\newline \small ovvero è il costo medio ponderato del capitale. \`E il tasso che una società si aspetta di pagare in media ai suoi investitori per poter ripagare il capitale prestato da quest`ultimi per acquistare i propri asset \par}		
		   
\end{multicols}
\end{acronym}