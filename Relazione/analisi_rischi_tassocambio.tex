L'impatto della variazione del tasso di cambio sull'analisi dei flussi mensili è stato valutato, considerando il tasso medio giornaliero, esaminando un campione dei tassi dal 07/07/2016 al 02/01/2017 (nell'analisi proposta è stato preso come riferimento il tasso di cambio del 15/12/2016 pari a $ \mbox{\ 1\: \euro \: = 136,51 ALL} $ evidenziato in \textbf{\textcolor{gray}{grigio}} nella tabella seguente).
\newline
\begin{longtable}{p{3cm}D{,}{,}{5.2}}
% didascalia ed etichetta
\caption{Andamento Tasso di Cambio (\euro -LEK)}\\
% intestazione iniziale
\toprule
	\makebox[3cm][c]{\textbf{Data}} & \multicolumn{1}{c}{\textbf{LEK}} \\
\midrule
\label{cambio_euro_lek}
\endfirsthead
% intestazione normale
\multicolumn{2}{l}{\footnotesize\itshape\tablename~\thetable:
continua dalla pagina precedente} \\
\toprule
	\makebox[3cm][c]{\textbf{Data}} & \multicolumn{1}{c}{\textbf{LEK}} \\
\midrule
\endhead
% piede normale
\midrule
\multicolumn{2}{r}{\footnotesize\itshape\tablename~\thetable:
continua nella prossima pagina} \\
\endfoot
% piede finale
\bottomrule
\multicolumn{2}{r}{\footnotesize\itshape\tablename~\thetable:
si conclude dalla pagina precedente} \\
\endlastfoot
% corpo della tabella
	\makebox[3cm][c]{$07/07/2016$} & 136,78\\
	\makebox[3cm][c]{$08/07/2016$} & 136,54\\	
 	\makebox[3cm][c]{$10/07/2016$} & 136,55\\
	\makebox[3cm][c]{$11/07/2016$} & 136,68\\
	\makebox[3cm][c]{$12/07/2016$} & 136,73\\	
 	\makebox[3cm][c]{$13/07/2016$} & 136,83\\
 	\makebox[3cm][c]{$14/07/2016$} & 136,77\\
	\makebox[3cm][c]{$15/07/2016$} & 136,52\\	
 	\makebox[3cm][c]{$17/07/2016$} & 136,53\\
	\makebox[3cm][c]{$18/07/2016$} & 136,57\\	
 	\makebox[3cm][c]{$19/07/2016$} & 136,43\\
 	\makebox[3cm][c]{$20/07/2016$} & 136,40\\
	\makebox[3cm][c]{$21/07/2016$} & 135,95\\	
 	\makebox[3cm][c]{$22/07/2016$} & 135,18\\
 	\makebox[3cm][c]{$24/07/2016$} & 135,17\\
	\makebox[3cm][c]{$25/07/2016$} & 135,98\\	
 	\makebox[3cm][c]{$26/07/2016$} & 136,34\\
 	\makebox[3cm][c]{$27/07/2016$} & 137,15\\
	\makebox[3cm][c]{$28/07/2016$} & 136,18\\	
 	\makebox[3cm][c]{$29/07/2016$} & 138,62\\
  	\makebox[3cm][c]{$31/07/2016$} & 136,12\\
	\makebox[3cm][c]{$01/08/2016$} & 136,11\\	
 	\makebox[3cm][c]{$02/08/2016$} & 136,80\\	
	\makebox[3cm][c]{$03/08/2016$} & 138,34\\	
	\makebox[3cm][c]{$04/08/2016$} & 136,06\\	
	\makebox[3cm][c]{$05/08/2016$} & 135,52\\	
	\makebox[3cm][c]{$07/08/2016$} & 135,55\\	
	\makebox[3cm][c]{$08/08/2016$} & 135,48\\	
	\makebox[3cm][c]{$09/08/2016$} & 136,21\\	
	\makebox[3cm][c]{$10/08/2016$} & 135,95\\	
	\makebox[3cm][c]{$11/08/2016$} & 136,01\\	
	\makebox[3cm][c]{$12/08/2016$} & 136,17\\	
	\makebox[3cm][c]{$14/08/2016$} & 136,13\\	
	\makebox[3cm][c]{$15/08/2016$} & 136,33\\	
	\makebox[3cm][c]{$16/08/2016$} & 135,77\\	
	\makebox[3cm][c]{$17/08/2016$} & 136,30\\	
	\makebox[3cm][c]{$18/08/2016$} & 137,37\\	
	\makebox[3cm][c]{$19/08/2016$} & 137,26\\	
	\makebox[3cm][c]{$21/08/2016$} & 136,62\\	
	\makebox[3cm][c]{$22/08/2016$} & 136,63\\	
	\makebox[3cm][c]{$23/08/2016$} & 136,86\\	
	\makebox[3cm][c]{$24/08/2016$} & 136,43\\	
	\makebox[3cm][c]{$25/08/2016$} & 137,22\\	
	\makebox[3cm][c]{$26/08/2016$} & 135,97\\	
	\makebox[3cm][c]{$28/08/2016$} & 135,77\\	
	\makebox[3cm][c]{$29/08/2016$} & 137,18\\	
	\makebox[3cm][c]{$30/08/2016$} & 136,64\\	
	\makebox[3cm][c]{$31/08/2016$} & 137,69\\
	\makebox[3cm][c]{$01/09/2016$} & 137,54\\
	\makebox[3cm][c]{$02/09/2016$} & 137,06\\
	\makebox[3cm][c]{$04/09/2016$} & 137,05\\
	\makebox[3cm][c]{$05/09/2016$} & 137,35\\
	\makebox[3cm][c]{$06/09/2016$} & 138,64\\
	\makebox[3cm][c]{$07/09/2016$} & 137,62\\
	\makebox[3cm][c]{$08/09/2016$} & 137,28\\
	\makebox[3cm][c]{$09/09/2016$} & 137,16\\
	\makebox[3cm][c]{$11/09/2016$} & 137,17\\
	\makebox[3cm][c]{$12/09/2016$} & 137,37\\
	\makebox[3cm][c]{$13/09/2016$} & 137,63\\
	\makebox[3cm][c]{$14/09/2016$} & 138,04\\
	\makebox[3cm][c]{$15/09/2016$} & 137,34\\
	\makebox[3cm][c]{$16/09/2016$} & 136,21\\
	\makebox[3cm][c]{$18/09/2016$} & 136,21\\
	\makebox[3cm][c]{$19/09/2016$} & 137,14\\
	\makebox[3cm][c]{$20/09/2016$} & 137,17\\
	\makebox[3cm][c]{$21/09/2016$} & 137,67\\
	\makebox[3cm][c]{$22/09/2016$} & 137,84\\
	\makebox[3cm][c]{$23/09/2016$} & 137,04\\
	\makebox[3cm][c]{$25/09/2016$} & 137,05\\
	\makebox[3cm][c]{$26/09/2016$} & 137,42\\
	\makebox[3cm][c]{$27/09/2016$} & 137,19\\
	\makebox[3cm][c]{$28/09/2016$} & 137,01\\
	\makebox[3cm][c]{$29/09/2016$} & 137,47\\
	\makebox[3cm][c]{$30/09/2016$} & 137,48\\
	\makebox[3cm][c]{$02/10/2016$} & 137,26\\
	\makebox[3cm][c]{$03/10/2016$} & 136,93\\
	\makebox[3cm][c]{$04/10/2016$} & 137,40\\
	\makebox[3cm][c]{$05/10/2016$} & 137,24\\
	\makebox[3cm][c]{$06/10/2016$} & 136,90\\
	\makebox[3cm][c]{$07/10/2016$} & 138,20\\
	\makebox[3cm][c]{$09/10/2016$} & 137,95\\
	\makebox[3cm][c]{$10/10/2016$} & 137,07\\
	\makebox[3cm][c]{$11/10/2016$} & 136,03\\
	\makebox[3cm][c]{$12/10/2016$} & 136,92\\
	\makebox[3cm][c]{$13/10/2016$} & 137,00\\
	\makebox[3cm][c]{$14/10/2016$} & 136,05\\
	\makebox[3cm][c]{$16/10/2016$} & 136,03\\
	\makebox[3cm][c]{$17/10/2016$} & 137,32\\
	\makebox[3cm][c]{$18/10/2016$} & 137,09\\
	\makebox[3cm][c]{$19/10/2016$} & 137,04\\
	\makebox[3cm][c]{$20/10/2016$} & 135,88\\
	\makebox[3cm][c]{$21/10/2016$} & 135,53\\
	\makebox[3cm][c]{$23/10/2016$} & 136,07\\
	\makebox[3cm][c]{$24/10/2016$} & 136,03\\
	\makebox[3cm][c]{$25/10/2016$} & 136,23\\
	\makebox[3cm][c]{$26/10/2016$} & 136,57\\
	\makebox[3cm][c]{$27/10/2016$} & 136,42\\
	\makebox[3cm][c]{$28/10/2016$} & 137,42\\
	\makebox[3cm][c]{$30/10/2016$} & 136,54\\
	\makebox[3cm][c]{$31/10/2016$} & 137,04\\	
	\makebox[3cm][c]{$01/11/2016$} & 137,48\\
	\makebox[3cm][c]{$02/11/2016$} & 137,06\\
	\makebox[3cm][c]{$03/11/2016$} & 136,47\\
	\makebox[3cm][c]{$04/11/2016$} & 137,19\\
	\makebox[3cm][c]{$06/11/2016$} & 136,41\\
	\makebox[3cm][c]{$07/11/2016$} & 136,66\\
	\makebox[3cm][c]{$08/11/2016$} & 136,57\\
	\makebox[3cm][c]{$09/11/2016$} & 134,87\\
	\makebox[3cm][c]{$10/11/2016$} & 136,32\\
	\makebox[3cm][c]{$11/11/2016$} & 135,98\\
	\makebox[3cm][c]{$13/11/2016$} & 136,00\\
	\makebox[3cm][c]{$14/11/2016$} & 136,00\\
	\makebox[3cm][c]{$15/11/2016$} & 135,89\\
	\makebox[3cm][c]{$16/11/2016$} & 136,05\\
	\makebox[3cm][c]{$17/11/2016$} & 134,59\\
	\makebox[3cm][c]{$18/11/2016$} & 135,38\\
	\makebox[3cm][c]{$20/11/2016$} & 135,38\\
	\makebox[3cm][c]{$21/11/2016$} & 135,80\\
	\makebox[3cm][c]{$22/11/2016$} & 135,80\\
	\makebox[3cm][c]{$23/11/2016$} & 135,94\\
	\makebox[3cm][c]{$24/11/2016$} & 135,81\\
	\makebox[3cm][c]{$25/11/2016$} & 136,11\\
	\makebox[3cm][c]{$27/11/2016$} & 136,19\\
	\makebox[3cm][c]{$28/11/2016$} & 135,73\\
	\makebox[3cm][c]{$29/11/2016$} & 136,37\\
	\makebox[3cm][c]{$30/11/2016$} & 135,68\\
	\makebox[3cm][c]{$01/12/2016$} & 135,77\\
	\makebox[3cm][c]{$02/12/2016$} & 136,05\\
	\makebox[3cm][c]{$04/12/2016$} & 134,45\\
	\makebox[3cm][c]{$05/12/2016$} & 136,84\\
	\makebox[3cm][c]{$06/12/2016$} & 137,10\\
	\makebox[3cm][c]{$07/12/2016$} & 135,81\\
	\makebox[3cm][c]{$08/12/2016$} & 135,78\\
	\makebox[3cm][c]{$09/12/2016$} & 135,86\\
	\makebox[3cm][c]{$11/12/2016$} & 135,65\\
	\makebox[3cm][c]{$12/12/2016$} & 135,81\\
	\makebox[3cm][c]{$13/12/2016$} & 135,63\\
	\makebox[3cm][c]{$14/12/2016$} & 134,65\\
	\rowcolor[gray]{.7}\makebox[3cm][c]{$15/12/2016$} & 136,51\\
	\makebox[3cm][c]{$16/12/2016$} & 135,73\\
	\makebox[3cm][c]{$18/12/2016$} & 135,66\\
	\makebox[3cm][c]{$19/12/2016$} & 135,17\\
	\makebox[3cm][c]{$20/12/2016$} & 134,45\\
	\makebox[3cm][c]{$21/12/2016$} & 134,37\\
	\makebox[3cm][c]{$22/12/2016$} & 134,43\\
	\makebox[3cm][c]{$23/12/2016$} & 134,35\\
	\makebox[3cm][c]{$25/12/2016$} & 134,60\\	
	\makebox[3cm][c]{$26/12/2016$} & 134,38\\
	\makebox[3cm][c]{$27/12/2016$} & 134,52\\
	\makebox[3cm][c]{$28/12/2016$} & 134,94\\
	\makebox[3cm][c]{$29/12/2016$} & 134,86\\
	\makebox[3cm][c]{$30/12/2016$} & 134,90\\
	\makebox[3cm][c]{$01/01/2017$} & 134,95\\
	\makebox[3cm][c]{$02/01/2017$} & 134,39\\				
\end{longtable}
Le principali grandezze che lo caratterizzano sono pari a (per approfondimenti si rimanda all'appendice \ref{sec:stimatori_tasso_cambio}):

\begin{savenotes}
\begin{table}[htb]
\centering
 \caption{Grandezze}
 \label{table:grandezze_cambio}
 \begin{tabular}{p{5cm}D{,}{,}{5.5}}
 \toprule
 	\multicolumn{1}{c}{\textbf{Grandezza}} & \multicolumn{1}{c}{\textbf{Valore}} \\
 \midrule 		
	\makebox[5cm][r]{Media Campionaria} & 136,47083\\
 	\makebox[5cm][r]{Varianza Campionaria Corretta} & 0,88366\\
 	\makebox[5cm][r]{Deviazione Standard Corretta} & 0,94003\\	
 \bottomrule
 \end{tabular} 
\end{table}
\end{savenotes}
L'intervallo di confidenza, di livello $0,9$ ad esse associato è pari a:
\begin{equation}
	\label{eq:intervallo_confidenza_tasso_cambio}
	\begin{split}
		\left [ 135,53 ; 137,41 \right]	 
	\end{split}
\end{equation}

\subsection[Casi di Studio]{Casi di Studio}
Si è analizzato, nel dettaglio, l'impatto che può presentare una variazione del tasso di cambio nel caso realistico in cui i dipendenti si possano assentare per malattia, in maniera uniforme durante l'arco dell'anno lavorativo (non si sono considerati i casi limite, ovvero la situazione per cui gli assenti per malattia si concentrano nei mesi di Gennaio e Dicembre). \newline Si è studiato, in particolare, il caso in cui il \emph{lek} possa subire, rispetto all'\emph{euro}:
\begin{itemize}
\item un \textbf{rafforzamento}, quindi si dovranno spendere \underline{più} \textit{euro} per acquistare la stessa quantità di \textit{lek} (\textbf{CASO PEGGIORE});
\item un \textbf{deprezzamento}, viceversa si dovranno utilizzare \underline{meno} \textit{euro} per acquistare la stessa quantità di \textit{lek} (\textbf{CASO MIGLIORE});
\end{itemize} 
Per tener conto di questo fenomeno si introduce l'elemento adimensionale $ f $ ($fattore \thinspace di \thinspace aggiustamento$) definito come il rapporto tra un tasso di cambio \emph{euro-lek} di riferimento fisso $x$ ed uno variabile $y$:
\begin{equation}
\label{eq:fattore_aggiustamento}
\begin{split}
	f = \frac{x}{y}
\end{split}
\end{equation}
questo fattore terrà conto, di conseguenza delle variazioni del tasso di cambio rispetto ad uno ben definito.
Nell'analisi proposta il valore di riferimento $x$, è quello del tasso di cambio al 15/12/2016:
\begin{equation}
\label{eq:fattore_aggiustamento_rif}
\begin{split}
	x = 136,51
\end{split}
\end{equation}
A titolo di esempio si consideri la seguente sezione della tabella (\ref{cambio_euro_lek}):
\begin{savenotes}
\begin{table}[htb]
\centering
 \caption{Tasso di Cambio (\euro -LEK)}
 \begin{tabular}{p{5cm}D{,}{,}{5.5}}
 \toprule
 	\multicolumn{1}{c}{\textbf{Giorno}} & \multicolumn{1}{c}{\textbf{Valore Cambio}} \\
 \midrule 		
	\makebox[3cm][c]{$06/12/2016$} & 137,10\\
	\makebox[3cm][c]{$07/12/2016$} & 135,81\\
	\makebox[3cm][c]{$08/12/2016$} & 135,78\\
	\makebox[3cm][c]{$09/12/2016$} & 135,86\\
	\makebox[3cm][c]{$11/12/2016$} & 135,65\\
	\makebox[3cm][c]{$12/12/2016$} & 135,81\\
	\makebox[3cm][c]{$13/12/2016$} & 135,63\\
	\makebox[3cm][c]{$14/12/2016$} & 134,65\\
	\rowcolor[gray]{.7}\makebox[3cm][c]{$15/12/2016$} & 136,51\\
	\makebox[3cm][c]{$16/12/2016$} & 135,73\\
	\makebox[3cm][c]{$18/12/2016$} & 135,66\\
	\makebox[3cm][c]{$19/12/2016$} & 135,17\\
	\makebox[3cm][c]{$20/12/2016$} & 134,45\\
	\makebox[3cm][c]{$21/12/2016$} & 134,37\\
 \bottomrule
 \end{tabular} 
\end{table}
\end{savenotes}
\newline
\newline
Il \emph{fattore di aggiustamento} calcolato con il tasso del $06/12/2016$ è pari a:
\begin{equation}
\label{eq:example_1_fattore_aggiustamento}
\begin{split}
	f = \frac{tasso \thinspace cambio \thinspace 15/12/2016}{tasso \thinspace cambio \thinspace 06/12/2016} = \frac{136,51}{137,10} = 0,9956
\end{split}
\end{equation}

calcolando il prodotto tra (\ref{eq:example_1_fattore_aggiustamento}) ed una certa quantità di \emph{euro}, ad esempio $ z = \mbox{\euro \thinspace 100}$:

\begin{eqnarray}
\label{eq:euro_example1}
	\gamma _1 (\mbox{\euro})& = & 0,9956 \cdot z 	\nonumber \\
			  & = & 0,9956 \cdot 100,00 \thinspace \mbox{\euro} \nonumber \\
			  & = & 99,56 \thinspace \mbox{\euro}
\end{eqnarray}
si ottiene il valore di $z = \mbox{\euro \thinspace 100}$ con un cambio \underline{favorevole} rispetto a quello di riferimento (ovvero quello del $15/12/2016$). A parità di prezzo in lek, quindi, dovremmo spendere meno \textit{euro} per acquistare un certo bene.\newline
Analogamente, ripetendo i calcoli precedenti con il tasso del $20/12/2016$:
\begin{equation}
\label{eq:example_2_fattore_aggiustamento}
\begin{split}
	f = \frac{tasso \thinspace cambio \thinspace 15/12/2016}{tasso \thinspace cambio \thinspace 20/12/2016} = \frac{136,51}{134,45} = 1,015
\end{split}
\end{equation}
\begin{eqnarray}
\label{eq:euro_example2}
	\gamma _2 (\mbox{\euro})& = & 1,015 \cdot z 	\nonumber \\
			  & = & 1,015 \cdot 100,00 \thinspace \mbox{\euro} \nonumber \\
			  & = & 101,50 \thinspace \mbox{\euro}
\end{eqnarray}
La formula del calcolo del VAN dovrà tener conto di queste variazioni, pertanto la (\ref{eq:van_caso_studio_2}) assumerà la seguente forma:
	\begin{eqnarray}
	\label{eq:van_caso_studio_tassocambio}
 		y(x) & = & - 56\thinspace 170,20 + \sum_{k=1}^{12} \frac{x - 60\thinspace 131,27 \cdot f}{(1+0,058)^k} \nonumber \\
 		 & = & -56\thinspace 170,20 + (x - 60\thinspace 131,27 \cdot f) \cdot \underbrace{\sum_{k=1}^{12} \frac{1} {(1+0,058)^{k}}}_{{}=8,47654} \nonumber \\
 		 & = & -56\thinspace 170,20 + (x - 60\thinspace 131,27 \cdot f) \cdot 8,47654		
	\end{eqnarray}  		

	\begin{tcolorbox}[colframe=blue!75!black,adjusted title=\textbf{Osservazione!}]
		Nella formula (\ref{eq:van_caso_studio_tassocambio}) si è considerato la variazione del tasso \emph{euro}-\emph{lek} soltanto per gli \emph{\ac{OPEX}}. \newline Rispetto a (\ref{eq:van_caso_studio_2}) il \emph{peso} quest'ultimi sono modificati in:
		\[ 60\thinspace 131,27 \cdot f \] 
	In linea teorica anche i \emph{\ac{CAPEX}} e le entrate mensili nette \emph{x} dovranno tener conto di questo fattore. Nel caso analizzato, ciò non si verifica, perchè, per quanto riguarda le entrate \emph{x}, Sistelia retribuisce direttamente in \emph{euro}, pertanto questa quantità non è influita dal cambio.\newline Un discorso analogo riguarda i \emph{\ac{CAPEX}}, in quanto i prezzi delle attrezzature sono fornite da Sistelia e sono forniti anch'essi in \emph{euro}.
	\end{tcolorbox} 
\subsubsection[Caso Peggiore - Deprezzamento dell'euro]{Caso Peggiore - Deprezzamento dell'euro}
\label{sec:cambio_favorevole}
La situazione peggiore si presenta, quando l'\emph{euro} vale $135,53 \thinspace lek$, ovvero quando il tasso \emph{euro-lek} presenta un valore pari al limite sinistro dell'intervallo di confidenza (\ref{eq:intervallo_confidenza_tasso_cambio}). Il tasso di cambio risulterebbe sfavorevole rispetto al caso medio (ovvero alla \textit{media campionaria} (\ref{table:grandezze_cambio})) in cui, invece, l'euro è scambiato con $136,47 \thinspace lek$.\newline
In queste condizioni il valore del \textit{fattore di aggiustamento}  (\ref{eq:fattore_aggiustamento}) è uguale a:
\begin{equation}
\label{eq:fattore_aggiustamento_caso_migliore}
\begin{split}
f = \frac{136,47}{135,53} = 1,007
\end{split}
\end{equation}
quindi la formula del \emph{\ac{VAN}} (\ref{eq:van_caso_studio_tassocambio}) diventa:

\begin{eqnarray}
\label{eq:van_caso_migliore}
 		y(x) & = & -56\thinspace 170,20 + (x - 60\thinspace 131,27 \cdot f) \cdot 8,47654 \nonumber \\
 			 & = & -56\thinspace 170,20 + (x - 60\thinspace 131,27 \cdot 1,007) \cdot 8,47654 \nonumber \\
 			 & = & -569\thinspace 443,25 + x \cdot 8,47654
\end{eqnarray}

Il valore di (\ref{eq:van_caso_migliore}) in corrispondenza del flusso di cassa mensile $ x = \mbox{73\thinspace 925,12 \euro}$ (\ref{table:van_malati_uniforme_anno}):
\begin{eqnarray}
\label{eq:van_flusso_cassa_malati_uniforme_caso_migliore}
 		y(x) & = & -569\thinspace 443,25 + x \cdot 8,47654 \nonumber \\
 			 & = & -569\thinspace 443,25 + 73\thinspace 925,12 \cdot 8,47654 \nonumber \\
 			 & = & 57\thinspace 185,985 \simeq 57\thinspace 185,99
\end{eqnarray}

\subsubsection[Caso Migliore - Rafforzamento dell'euro]{Caso Migliore - Rafforzamento dell'euro}
\label{sec:cambio_sfavorevole}
La situazione migliore si presenta, invece, quando l'\emph{euro} vale $137,41 \thinspace lek$, ovvero quando il tasso \emph{euro-lek} presenta un valore pari al limite destro dell'intervallo di confidenza (\ref{eq:intervallo_confidenza_tasso_cambio}). Il tasso di cambio risulterebbe favorevole rispetto al caso medio (ovvero alla \textit{media campionaria} (\ref{table:grandezze_cambio})) in cui, invece, l'euro è scambiato con $136,47 \thinspace lek$.\newline
In queste condizioni il valore del \textit{fattore di aggiustamento}  (\ref{eq:fattore_aggiustamento}) è uguale a:
\begin{equation}
\label{eq:fattore_aggiustamento_caso_peggiore}
\begin{split}
f = \frac{136,47}{137,41} = 0,993
\end{split}
\end{equation}
quindi la formula del \emph{\ac{VAN}} (\ref{eq:van_caso_studio_tassocambio}) diventa:

\begin{eqnarray}
\label{eq:van_caso_peggiore}
 		y(x) & = & -56\thinspace 170,20 + (x - 60\thinspace 131,27 \cdot f) \cdot 8,47654 \nonumber \\
 			 & = & -56\thinspace 170,20 + (x - 60\thinspace 131,27 \cdot 0,993) \cdot 8,47654 \nonumber \\
 			 & = & -562\thinspace 307,38 + x \cdot 8,47654
\end{eqnarray}

Il valore di (\ref{eq:van_caso_peggiore}) in corrispondenza del flusso di cassa mensile $ x = \mbox{73\thinspace 925,12 \euro}$ (\ref{table:van_malati_uniforme_anno}):
\begin{eqnarray}
\label{eq:van_flusso_cassa_malati_uniforme_caso_peggiore}
 		y(x) & = & -562\thinspace 307,38 + x \cdot 8,47654 \nonumber \\
 			 & = & -562\thinspace 307,38 + 73\thinspace 925,12 \cdot 8,47654 \nonumber \\
 			 & = & 64\thinspace 321,8567 \simeq 64\thinspace 321,86
\end{eqnarray}

\clearpage
\section[Sintesi Risultati]{Sintesi Risultati}
I risultati ottenuti in (\ref{sec:cambio_favorevole}) e in (\ref{sec:cambio_sfavorevole}) possono essere riassunti nella seguente tabella:

\begin{savenotes}
\begin{table}[htb]
\centering
 \caption{Variazione VAN (Caso di studio 15 \% comprensivo del rischio malattie)}
 \begin{tabular}{D{,}{,}{5.2}D{,}{,}{4.3}D{,}{,}{7.2}}
 \toprule
 	\multicolumn{1}{c}{\textbf{Cambio \emph{euro-lek}}} & 
 	\multicolumn{1}{c}{\textbf{Aggiustamento}}			&
 	\multicolumn{1}{c}{\textbf{Valore VAN (\euro)}} \\
 \midrule 		
	135,53 & 1,007 & 57\thinspace 185,99 \\
	137,41 & 0,993 & 64\thinspace 321,86 \\	
 \bottomrule
 \end{tabular} 
\end{table}
\end{savenotes}