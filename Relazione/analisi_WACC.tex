\section[WACC]{WACC}
	\begin{equation}
	\label{eq:wacc}
	\begin{split}
		WACC = \frac{D}{D+E} \cdot K_d + \frac{E}{D+E} \cdot K_e 
	\end{split}
	\end{equation}
	La formula \ref{eq:wacc}, però non tiene conto della quota di imposte che gravano sulla quota da restituire (cioè sul valore di \textbf{D}). \newline Se definiamo con \textbf{t}, il valore della quota di imposte che gravano su \textbf{D} ( nel nostro caso $ t = 0,15 $ ), allora la \ref{eq:wacc} diventa:
	\begin{equation}
	\label{eq:wacc_tax}
	\begin{split}
		WACC = \frac{D}{D+E} \cdot K_d \cdot ( 1 - t ) + \frac{E}{D+E} \cdot K_e 
	\end{split}
	\end{equation}	
	Il nostro caso di studio prevedeva di richiedere un finanziamento di $\mbox{\euro} \: 30\thinspace 000,00$ presso la \textbf{\ac{Fibank Albania}} in modo tale da essere in grado di fronteggiare le spese di installazione dei vari impianti in quanto la quota iniziale a disposizione dei soci fondatori non era sufficiente, oltre a questo finanziamento non abbiamo previsto altre fonti di capitale, pertanto la quota $K_e = 0$.
\newline
La \textbf{\ac{Fibank Albania}} prevede, in particolare una quota di interesse pari al 4,4 \% della quota del finanziamento, quindi in questo caso dovremmo restituire circa:
	\begin{equation}
	\label{eq:interessi_fibank}
	\begin{split}
		quota \: interessi = 0,044 \cdot 30\thinspace 000 = 1\thinspace 320 \:\mbox{\euro} 
	\end{split}
	\end{equation}	
Il nostro valore del \ac{WACC}, quindi, sarà pari a:
	\begin{equation}
	\label{eq:wacc_tax_value}
	\begin{split}
		WACC = \frac{30\thinspace 000}{65\thinspace 000} \cdot 0,044 \cdot ( 1 - 0,15 ) = 0,01726 
	\end{split}
	\end{equation}