\section[WACC]{WACC}
	Formalmente il \ac{WACC} è definito come:	
	\begin{equation}
	\label{eq:wacc}
	\begin{split}
		WACC = \frac{D}{D+E} \cdot K_d + \frac{E}{D+E} \cdot K_e 
	\end{split}
	\end{equation}
	La formula \ref{eq:wacc}, però non tiene conto della quota di imposte che gravano sulla quota da restituire (cioè sul valore di \textbf{D}). \newline 
	Se definiamo con \textbf{t}, il valore della quota di imposte che gravano su \textbf{D} ( nel nostro caso $ t = 0,20 $ ), allora la \ref{eq:wacc} diventa:
	\begin{equation}
	\label{eq:wacc_tax}
	\begin{split}
		WACC = \frac{D}{D+E} \cdot K_d \cdot ( 1 - t ) + \frac{E}{D+E} \cdot K_e 
	\end{split}
	\end{equation}
	Il valore $K_e$, ovvero il \textbf{costo del capitale proprio} (\textit{equity}) è stato considerato pari a  quello assunto dall'azienda italiana \textbf{Telecom Italia Spa} il giorno Martedì 24/01/2017\cite{wacc_ke}:
	\begin{equation}
	\label{eq:ke_telecom}
	\begin{split}
		K_e = 0,07774
	\end{split}
	\end{equation} 
		
	\begin{tcolorbox}[colframe=blue!75!black,adjusted title=\textbf{Osservazione!}]
		Nel calcolo del valore di $K_e$ si è assunto che i tre soci si comportino come se costituissero un unico proprietario, quindi è come se ci fosse un unico investimento iniziale eseguito da una sola persona fisica su cui è determinato il costo relativo. In realtà si può sperimentare che i tre soci possano avere dei costi $K_e$ differenti.
	\end{tcolorbox}	
	
	A fronte del finanziamento di $\mbox{\euro} \: 85\thinspace 000,00$ richiesto, per fronteggiare le spese di installazione dei vari impianti, il valore del parametro $K_d$, che tiene conto del \textbf{costo del debito} è pari al tasso d'interesse applicato dalla \ac{Fibank Albania}, ovvero il $4,4 \%$, quindi:  
	\begin{equation}
	\label{eq:kd_fibank}
	\begin{split}
		K_e = 0,044
	\end{split}
	\end{equation} 	
	 il valore del \ac{WACC} è pari a:
	\begin{eqnarray}
	\label{eq:wacc_tax_value}
		\mbox{WACC} & = & \frac{85\thinspace 000}{180\thinspace 000} \cdot 0,044 \cdot ( 1 - 0,20 ) + \frac{95\thinspace 000}{180\thinspace 000} \cdot 0,07774 \nonumber \\
		\nonumber \\
			 & = & 0,0576517 \simeq 0,058
	\end{eqnarray}
	Infine, il valore del \ac{ROD} relativo, definito come:
	\begin{equation}
	\label{eq:rod_definizione}
	\begin{split}
		ROD = \frac{\textit{Oneri finanziari}}{\textit{Capitale di debito}} 
	\end{split}
	\end{equation}
	è pari a:
	\begin{eqnarray}
	\label{eq:rod_calcolo}	
		ROD & = & \frac{0,044 \cdot D}{D} \nonumber \\
										   \nonumber \\
			& = & 0,044	
	\end{eqnarray}