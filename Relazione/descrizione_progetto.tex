\chapter{Descrizione Progetto}
\renewcommand{\thesection}{\arabic{section}}
Lo scopo di questo progetto consiste nella valutazione dei costi operativi di un call center con operatività 24 ore su 24, 7 giorni su 7 per conto di un'azienda del settore utilities. \newline
Nello specifico sono stati analizzati i costi sostenuti durante l'anno solare 2016 ( dal 1 Gennaio al 31 Dicembre ) da una società albanese, con sede nella capitale Tirana, che fornisce un servizio di \textbf{outbound} per conto della società \textbf{Sistelia Group S.r.l.}, specializzata nell'installazione di piattaforme di call center e fornitore di richieste avanzate per conto di aziende terze operanti nei più disparati settori. \newline
La società oggetto dello studio, la \textbf{\ac{ANV S.r.l.}}, costituita il 1 Gennaio 2016, ha un capitale sociale di partenza pari a \euro 35000 ripartito equamente tra i suoi 3 soci. La sua sede legale e sociale è stata stabilita in Albania perchè in questo modo si riescono a sfruttare le opportunità che offre questo paese per attrarre gli investimenti esteri, in particolare:
\begin{itemize}
\item una burocrazia snella ed un sistema fiscale che agevole tramite apposite normative le iniziative imprenditoriali.
		A titolo di esempio, si può evidenziare come la costituzione di una società a responsabilità limitata può avvenire nell'arco di un giorno e garantisce la registrazione anche presso l'Ufficio Imposte e delle Aziende;
\item un \textbf{cambio favorevole}. La moneta locale, il \textit{lek} (\textbf{ALL}), presenta il seguente tasso di cambio:
	\begin{center}
		1 \euro : 136,51 ALL\footnote{dati aggiornati al 15/12/2016 (fonte http://it.coinmill.com/ALL\_EUR.html)}
	\end{center}
	
	\begin{tcolorbox}[colframe=blue!75!black,adjusted title=\textbf{Osservazione!}]
		Per nostra semplicità abbiamo eseguito i nostri calcoli in \textit{euro} considerando dati espressi in LEK rappresentativi del tenore di vita a Tirana.
	\end{tcolorbox}

\item una \textbf{posizione geografica strategica} tra i paesi dell'\ac{UE} (Italia e Grecia) e quelli della penisola balcanica (confina con il Montenegro a nord-ovest, il Kosovo a nord-est, la Macedonia ad est ) che permette facilmente di poter espandere la propria presenza nei mercati di questi paesi, senza dimenticare altri potenziali paesi come la Croazia, la Romania o la Bulgaria.
\item la presenza di \textbf{accordi bilaterali} con l'Italia (che costituisce il principale partner commerciale) e con l'\ac{UE} in generale, che favoriscono gli scambi commerciali e, nel nostro caso, permettono di evitare la \textbf{doppia imposizione}\cite{accordialbaniaitalia}. In pratica, gli utili che realizzeremo in Albania andranno a costituire una base imponibile per il pagamento delle tasse soltanto in questo paese e non in Italia.  
\end{itemize}  
\section[Organigramma Aziendale]{Organigramma Aziendale}
La struttura della\textbf{ \ac{ANV S.r.l.}} prevede una struttura gerarchica piramidale, in particolare:
\begin{itemize}

 	\item i \textbf{soci fondatori} ricevono gli utili generati dalla società ripartiti in base alle quote possedute della stessa, adeguano il patrimonio societario in base alle strategie descritte nel piano di investimento annuale presentato dal \textbf{\ac{CEO}} e giudicano l'operato di quest'ultimo sui risultati ottenuti; 
 	
	\item un \textbf{presidente}, che ricopre anche il ruolo di \textbf{\ac{CEO}} responsabile degli investimenti, a capo del consiglio di amministrazione che prevede oltre ai soci fondatori anche altri 3 manager;
	
	\item \textbf{3 manager} responsabili, ognuno, del funzionamento di una squadra di 10 centralinisti;
	
	\item \textbf{30 centralinisti} suddivisi in due turni da 6 ore ciascuno in una giornata dalle 9:30 alle 21:30. 

\end{itemize}  

Tale struttura può essere schematizzata dalla seguente figura:
\newline


\begin{tikzpicture}[
  level 1/.style={sibling distance=55mm},
  edge from parent/.style={->,draw},
  >=latex]

% root of the the initial tree, level 1
\node[root] {CEO}
% The first level, as children of the initial tree
  child {node[level 2] (c1) {Manager \#1}}
  child {node[level 2] (c2) {Manager \#2}}
  child {node[level 2] (c3) {Manager \#3}};

% The second level, relatively positioned nodes
\begin{scope}[every node/.style={level 3}]
\node [below of = c1, xshift=30pt] (c11) {Centralinista \#1};
\node [below of = c11] (c12) {Centralinista ...};
\node [below of = c12] (c13) {Centralinista \#5};

\node [below of = c2, xshift=30pt] (c21) {Centralinista \#6};
\node [below of = c21] (c22) {Centralinista ...};
\node [below of = c22] (c23) {Centralinista \#10};

\node [below of = c3, xshift=30pt] (c31) {Centralinista \#11};
\node [below of = c31] (c32) {Centralinista ...};
\node [below of = c32] (c33) {Centralinista \#15};
\end{scope}

% lines from each level 1 node to every one of its "children"
\foreach \value in {1,2,3}
  \draw[->] (c1.195) |- (c1\value.west);

\foreach \value in {1,...,3}
  \draw[->] (c2.195) |- (c2\value.west);

\foreach \value in {1,...,3}
  \draw[->] (c3.195) |- (c3\value.west);
\end{tikzpicture}

Si può osservare come si tratta di una società di piccole dimensioni adeguata sia alle disponibilità economiche di ciascun socio sia al potenziale ufficio disponibile a Tirana, in quanto già provvisto della maggiorparte delle strutture necessarie al funzionamento di un call center.

\section[Sistema Fiscale]{Sistema Fiscale}
  
\subsection[Persone Giuridiche]{Persone Giuridiche}
Una persona giuridica, ovvero un ente il cui ordinamento giuridico attribuisce la \textit{capacità giuridica} (diventando, quindi, un \textbf{soggetto di diritto}) è considerata come residente in Albania se ha una struttura permanente, la sede principale, o una sede per la reale gestione degli affari nel Paese.
\subsubsection{Imposta sul reddito aziendale} 
Tutte le imprese che siano albanesi o straniere registrate ai fini \ac{IVA} sono soggette all'\textit{imposta sul reddito aziendale} calcolata sulla base delle seguenti aliquote:
\begin{itemize}
	\item \textbf{15\%}, per le grandi imprese;
	\item \textbf{imposta semplificata} per le piccole imprese o piccoli imprenditori che realizzano un fatturato annuo lordo inferiore di \textbf{ALL 8 milioni} (circa 58603,77\euro). Le aliquote previste sono:
		\begin{center}
 			\begin{tabular}{SS[table-comparator = true]}
 			\toprule 
 				{Aliquota Applicata (\%)} & {Fatturato Annuale(ALL)} \\
 			\midrule
 				5 & \numrange{5000000}{8000000} \\
 				0 & < 5000000 \\
 			\bottomrule
 			\end{tabular} 
		\end{center}
\end{itemize} 
\subsubsection{IVA}
E' applicata sulla vendita delle merci e dei servizi a un tasso standard del 20\% e 10\% sulle medicine. La VAT non si applica sulle
esportazioni e sui servizi internazionali come per esempio il trasporto di merci e passeggeri.
\subsection[Persone Fisiche]{Persone Fisiche}
Una persona fisica, invece, è soggetta al pagamento delle tasse relative ai guadagni realizzati all'interno del territorio albanese, se non è residente, altrimenti deve pagare le tasse su tutti i guadagni realizzati anche all'estero.
Sono previste le seguenti aliquote:\newline
\newpage
\begin{savenotes}
\begin{table}[htb]
	\centering
	\begin{tabular}{D{,}{,}{5.2}D{,}{,}{5.2}c}
 \toprule
 	\multicolumn{2}{c}{\textbf{Reddito da lavoro mensile (in ALL)}} & \textbf{Aliquota} \\
 	Da & Fino\ a & \\
 \midrule
 	0 & 30000 & 0\% \\
 	30001 & 130000& 13\% dell'importo superiore ad ALL 30000\\
 	130001 & \ & ALL 13000 + 23\% dell'importo superiore ad ALL 130000 \\
 \bottomrule
 \end{tabular} 
\end{table}
\end{savenotes}

per semplicità riportiamo la precedente tabella con i valori riportati in \textbf{euro}:

\begin{savenotes}
\begin{table}[htb]
	\centering
	\begin{tabular}{D{,}{,}{5.2}D{,}{,}{5.2}c}
 \toprule
 	\multicolumn{2}{c}{\textbf{Reddito da lavoro mensile (in \euro)}} & \textbf{Aliquota} \\
 	Da & Fino\ a & \\
 \midrule
 	0 & 219,77 & 0\% \\
 	219,77 & 952,31 & 13\% dell'importo superiore ad \euro \hspace{0,0150625cm} 219,77\\
 	952,31 & \ & \euro \hspace{0,0150625cm} 95,23 + 23\% dell'importo superiore ad \euro \hspace{0,0150625cm} 952,31 \\
 \bottomrule
 \end{tabular} 
\end{table}
\end{savenotes}




Nel nostro caso avremmo la seguente situazione:

\begin{savenotes}
\begin{table}[htb]
\centering
 \caption{Stipendi Dipendenti}
 \begin{tabular}{rD{,}{,}{5.2}D{,}{,}{5.2}D{,}{,}{5.2}}
 \toprule
 	& \multicolumn{1}{c}{Centralinista} & \multicolumn{1}{c}{Manager} & \multicolumn{1}{c}{CEO} \\
 \midrule
 	Reddito Imponibile Mensile (\euro)& 459,67 & 947,90 & 5163,83 \\ 
 	Imposta sui Redditi (\euro)& 31,19\footnote{(aliquota del 13,00 \%) pari a (459,67-219,77)*0,13} & 94,66\footnote{(aliquota del 13,00 \%) pari a (947,90-219,77)*0,13} & 1063,88 \footnote{(aliquota del 23,00 \%) pari a (5163,83-952,31)*0,23+95,23}\\
	Contributi Previdenziali (11,20 \%)(\euro) & 51,48 & 106,16 & 578,35 \\
	Stipendio Netto (\euro) & 377,00 & 747,08 & 3521,60 \\ 	
 \bottomrule
 \end{tabular} 
\end{table}
\end{savenotes}

\begin{savenotes}
\begin{table}[htb]
\centering
 \caption{Costo Azienda Dipendenti}
 \begin{tabular}{rD{,}{,}{5.2}D{,}{,}{5.2}D{,}{,}{5.2}D{,}{,}{5.2}}
 \toprule
 	& \multicolumn{1}{c}{Centralinista} & \multicolumn{1}{c}{Manager} & \multicolumn{1}{c}{CEO} & \multicolumn{1}{c}{TOTALE} \\
 \midrule
 	Reddito Imponibile Mensile (\euro)& 459,67 & 947,90 & 5163,83 & \\ 
	Contributi Previdenziali (16,70 \%)(\euro) & 76,76 & 158,30 & 862,36 & \\
	\textbf{Costo Mensile Singolo Dipendente} (\euro) & 536,43 & 1106,20 & 6026,19 & \\ 	
	num. dipendenti & 30 & 3 & 1 & 34 \\
	\textbf{Costo Mensile Dipendenti} (\euro) & 16092,90 & 3318,60 & 6026,19 & 25437,69\\
	\textbf{Costo Annuale Dipendenti} (\euro) & 193114,80 & 39823,20 & 72314,28 & 305252,28\\ 	
 \bottomrule
 \end{tabular} 
\end{table}
\end{savenotes}

\section[Gruppo Sistelia]{Gruppo Sistelia}

\subsection[ReteTurismo]{ReteTurismo}