  \chapter{Calcolo Stimatori Malattie Dipendenti}
	Ci siamo serviti degli strumenti sopra descritti (variabile aleatoria \textbf{t-Student} e \textbf{intervalli di confidenza}) per stimare con un certo \textbf{livello di confidenza} quale fosse l'intervallo di definizione del nostro parametro di interesse (il numero di malattie per dipendente atteso nell'anno corrente). Più precisamente abbiamo dovuto far ricorso ad una \textbf{quantità pivotale} di distribuzione \textbf{t-Student} poichè la varianza teorica del campione preso in considerazione è incognita e pertanto la nostra stima intervallare si basa sulla \textbf{varianza campionaria corretta}. L'intervallo di confidenza inserito nel seguito è stato utilizzato con un valore del \textbf{quantile} della \textbf{t-Student} pari alla radice della numerosità del campione in esame ($\sqrt{23}$), assicurando un \textbf{livello di confidenza} molto elevato (approssimabile a \textbf{1}). S\ped{n} sta ad identificare la \textbf{varianza campionaria corretta}, $\bar{X\ped{n}}$ la \textbf{media campionaria}, $t\ped{1-\frac{\alpha}{2},n-1}$ il quantile della \textbf{t-Student} con n gradi di libertà per il quale sfruttiamo la simmetria della distribuzione.

\[
\{\bar{X\ped{n}} - \frac{S\ped{n}}{\sqrt{n}}t\ped{1-\frac{\alpha}{2},n-1} \leq \mu \leq \bar{X\ped{n}} + \frac{S\ped{n}}{\sqrt{n}}t\ped{1-\frac{\alpha}{2},n-1}\}
\]

  \label{sec:stimatori_malattie_dipendenti}
\begin{comment}

\begin{savenotes}
\begin{table}[htb]
\centering
 \caption{Statistiche}
 \begin{tabular}{p{5cm}D{,}{,}{5.2}D{,}{,}{5.2}}
 \toprule
 	 & \multicolumn{1}{c}{\textbf{x_i}} & \multicolumn{1}{c}{\textbf{x_i}} \\
 \midrule 		
 	 & 10,7 & \\
 	 & 11,1 & \\
 	 & 11,1 & \\
 	 & 11,4 & \\
 	 & 11,4 & \\
 	 & 11,6 & \\
	 & 11,5 & \\ 
	 & 11,3 & \\
	 & 11,2 & \\
	 & 11,5 & \\
	 & 11,6 & \\
	 & 11,8 & \\
	 & 12,1 & \\
	 & 12,2 & \\
	 & 11,8 & \\
	 & 11,4 & \\
	 & 11,4 & \\
	 & 11,6 & \\
	 & 11,7 & \\
	 & 11,6 & \\
	 & 11,6 & \\
	 & 11,7 & \\							  
	 & 11,8 & \\
	 & 11,8 & \\
 \midrule
 	\makebox[5cm][r]{Totale} & 265,5  & 0,62068\\	
 \bottomrule
 \end{tabular} 
\end{table}
\end{savenotes}   
  
   \begin{tabular}{SS[table-format=2]}
 \toprule
 	{Anno} & {Giorni Malattia} \\
 \midrule


 \bottomrule
 \end{tabular} 
 
 \end{comment}