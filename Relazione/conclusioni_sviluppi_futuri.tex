\chapter[Conclusioni]{Conclusioni}

Lo studio effettuato per la realizzazione e per l'analisi dei costi prodotti in un call center ha prodotto degli interessanti aspetti.
\newline
Per prima cosa abbiamo notato che, nel nostro caso di studio, i costi di tipo \textbf{\ac{OPEX}} sono predominanti rispetto ai \textbf{\ac{CAPEX}} e pertanto abbiamo cercato di sopperire a ciò ottimizzando il più possibile le strategie aziendali. A tal proposito abbiamo definito una stima dei guadagni rendendo più veritiero il nostro progetto, analizzando gli indicatori tipici ( \textbf{\ac{VAN}} ) per realizzare una stima preliminare della remuneratività dell'azienda.
\newline
L'analisi del \textbf{\ac{VAN}} è stata anche raffinata attraverso l'analisi dei rischi, i quali incidono sui guadagni in maniera più o meno evidente e possono rendere un progetto non più competitivo.
\newline
Grande rilevanza nella nostra analisi viene assunta dalla stima del tasso di successo per quanto concerne la sottoscrizione di eventuali contratti sull'intero volume di traffico prodotto, la quale influenza attivamente i futuri guadagni, rendendo il nostro progetto affetto da assunzioni teoriche che potrebbero non rispecchiare fedelmente la realtà. 