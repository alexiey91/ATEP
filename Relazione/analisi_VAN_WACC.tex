\chapter[Indicatori VAN-WACC-Punto di Pareggio]{Indicatori VAN-WACC-Punto di Pareggio}
\section[VAN]{VAN}
	\begin{equation}
	\label{eq:van}
	\begin{split}
 		w = \sum_{k=0}^n \frac{C_k}{(1+r)^k}
	\end{split}
	\end{equation}
	ci possiamo, quindi, calcolare il valore del \ac{TIR} corrispondente. Per definizione il \ac{TIR} è pari a:
	\begin{equation}
	\label{eq:tir}
	\begin{split}
 		\sum_{k=0}^n \frac{C_k}{(1+i)^k} = 0
	\end{split}
	\end{equation}	 
\section[WACC]{WACC}
	\begin{equation}
	\label{eq:wacc}
	\begin{split}
		WACC = \frac{D}{D+E} * K_d + \frac{E}{D+E} * K_e 
	\end{split}
	\end{equation}
\section[Analisi di Pareggio]{Analisi di Pareggio}
Si analizza, infine, il \textbf{punto di pareggio}, ovvero la quantità di chiamate necessarie per avere un fatturato tale da ricoprire l'investimento iniziale, in modo tale da chiudere il periodo di riferimento senza perdite né profitti.

Il \textbf{break even period} (periodo di pareggio), ovvero il periodo di tempo necessario per il recupero dell'esborso iniziale è quindi pari a:   	