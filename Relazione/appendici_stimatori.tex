  \chapter{Stimatori}
  \section{Media Campionaria}
	Dato un campione di $n$ variabili aleatorie indipendenti (se il campionamento fosse con ripetizioni o la popolazione di riferimento infinita) $ X_1 \ldots X_n $, si definisce \textbf{media campionaria}, la quantità:
	\begin{equation}
	\label{eq:media_campionaria}
	\begin{split}
		\bar{X} = \frac{1}{n} \cdot \sum_{i=0}^n X_i
	\end{split}
	\end{equation}	    

  \section{Varianza e Deviazione Standard Campionaria Corretta}
  Data la \ref{eq:media_campionaria} si può calcolare la \textbf{varianza} di $X$ attraverso la \textbf{varianza campionaria}: 
	\begin{equation}
	\label{eq:varianza_campionaria}
	\begin{split}
		\bar{S_n} = \frac{1}{n-1} \cdot \sum_{i=0}^n (X_i-\bar{X})^2
	\end{split}
	\end{equation}	      
  Si divide, in particolare, per $n-1$ (e non per $n$) perchè in questo modo:
	\begin{equation}
	\label{eq:stimatore_varianza_corretto}
	\begin{split}
		\mathbf{E}(S^2) = \mathbf{V}(X_i) = \sigma^2
	\end{split}
	\end{equation}	
  ossia, il valore $S^2 = \sigma^2$ che è la quantità che si vuole stimare.
  Uno stimatore che presenta questa proprietà si dice \textbf{non distorto} ( o \textbf{corretto} ).
  La quantità:
	\begin{equation}
	\label{eq:deviazione_standard_corretta}
	\begin{split}
		\sqrt{S^2} = \sqrt{\sigma^2} = \sigma
	\end{split}
	\end{equation}
  è definita, invece \textbf{deviazione standard corretta}.
  \section{Intervalli di Confidenza}   
  In contrapposizione alla stima puntuale di parametri di distribuzioni di probabilità, la stima intervallare permette di ipotizzare    con una certo \textbf{livello} di plausibilità che un dato parametro da stimare si trovi all'interno di un certo intervallo casuale (\textbf{V\ped{1}},\textbf{V\ped{2}}). Il valore di $1-\beta$ identifica proprio il \textbf{livello di confidenza} \eqref{eqn:conf}.
  
\begin{equation}
\label{eqn:conf}
Pr(V\ped{1} < \theta < V\ped{2}) = 1-\beta
\end{equation}    
