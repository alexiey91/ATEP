\chapter[Sistema Fiscale Albanese]{Sistema Fiscale Albanese}
  \label{sec:normative_fiscali_albania}

\section[Persone Fisiche]{Persone Fisiche}
\subsection[Imposta sui Redditi]{Imposta sui Redditi}
\label{sec:imposta_redditi_albania}
Una persona fisica, invece, è soggetta al pagamento delle tasse relative ai guadagni realizzati all'interno del territorio albanese, se non è residente, altrimenti deve pagare le tasse su tutti i guadagni realizzati anche all'estero.
Sono previste le seguenti aliquote:\newline
\begin{savenotes}
\begin{table}[htb]
	\centering
	\begin{tabular}{D{,}{,}{5.2}D{,}{,}{5.2}c}
 \toprule
 	\multicolumn{2}{c}{\textbf{Reddito da lavoro mensile (in ALL)}} & \textbf{Aliquota} \\
 	Da & Fino\ a & \\
 \midrule
 	0 & 30\thinspace 000 & 0\% \\
 	30\thinspace 001 & 130\thinspace 000& 13\% dell'importo superiore ad ALL 30\thinspace 000\\
 	130\thinspace 001 & \ & ALL 13\thinspace 000 + 23\% dell'importo superiore ad ALL 130\thinspace 000 \\
 \bottomrule
 \end{tabular} 
\end{table}
\end{savenotes}

per semplicità riportiamo la precedente tabella con i valori riportati in \textbf{euro}:

\begin{savenotes}
\begin{table}[htb]
	\centering
	\begin{tabular}{D{,}{,}{5.2}D{,}{,}{5.2}c}
 \toprule
 	\multicolumn{2}{c}{\textbf{Reddito da lavoro mensile (in \euro)}} & \textbf{Aliquota} \\
 	Da & Fino\ a & \\
 \midrule
 	0 & 219,77 & 0\% \\
 	219,77 & 952,31 & 13\% dell'importo superiore ad \euro \: 219,77\\
 	952,31 & \ & \euro \hspace{0,0150625cm} 95,23 + 23\% dell'importo superiore ad \euro \hspace{0,0150625cm} 952,31 \\
 \bottomrule
 \end{tabular} 
\end{table}
\end{savenotes}
\subsection[Previdenza Sociale]{Previdenza Sociale}
\label{sec:previdenza_sociale_albania}
In base alla legge \textbf{n. 7703 del 11/05/1993} \textit{"Sulla previdenza sociale nella Repubblica d'Albania"} e alla legge \textbf{n. 10383 del 24/02/2011} \textit{"Sulla previdenza obbligatoria sanitaria nella Repubblica d'Albania"} e successive modifiche, i datori di lavoro e i dipendenti sono tenuti a versare i contributi obbligatori per la previdenza sociale e sanitaria. \newline
Il datore di lavoro deve versare mensilmente , per ogni dipendente, all'\ac{ISSH}, agendo come \textbf{sostituto d'imposta}, una quota pari al \textit{27,9 \%} dello stipendio lordo percepito da ogni dipendente.
Della quota prevista, però:
\begin{itemize}
\item il \textbf{16,7 \%} è a carico del datore di lavoro;
\item il \textbf{11,2 \%} è a carico del dipendente;
\end{itemize}
In sintesi, quindi, dato lo stipendio dell'i-esimo dipendente:
\begin{equation}
\label{eq:stipendio_dip}
S_i
\end{equation}
la sua \underline{quota prevista} per la previdenza sociale è pari a:
\begin{equation}
\label{eq:quota_issh}
quota \: issh = 0,279 \cdot S_i
\end{equation}
ma la (\ref{eq:quota_issh}) \underline{non} è totalmente a carico del dipendente, ma dovrà contribuire solamente:
\begin{equation}
\label{eq:quota_issh_dip}
quota \: dip = 0,112 \cdot S_i
\end{equation}
mentre la rimanente parte dovrà essere corrisposta da parte dell'azienda:
\begin{equation}
\label{eq:quota_issh_dip_azienda}
quota \: dip \:azienda = 0,167 \cdot S_i
\end{equation}
ma la (\ref{eq:quota_issh_dip_azienda}) \underline{non verrà detratta} dallo stipendio del dipendente i-esimo, ma verrà pagata come quota esterna ad esso.

\subsection[Buste Paga Singolo Dipendente]{Buste Paga Singolo Dipendente}
Tenendo conto, quindi, delle imposte (\ref{sec:imposta_redditi_albania}) e (\ref{sec:previdenza_sociale_albania}), la busta paga dei dipendenti che ricoprono il ruolo di \textit{Centralinista}, \textit{Manager} e \textit{CEO} sarà costituita rispettivamente dalle seguenti voci: 
\begin{savenotes}
\begin{table}[htb]
\centering
 \caption{Busta Paga Dipendenti}
 \begin{tabular}{rD{,}{,}{5.2}D{,}{,}{5.2}D{,}{,}{5.2}}
 \toprule
 	& \multicolumn{1}{c}{Centralinista} & \multicolumn{1}{c}{Manager} & \multicolumn{1}{c}{CEO} \\
 \midrule
 	Reddito Imponibile Mensile (\euro)& 459,67 & 947,90 & 5\thinspace 163,83 \\ 
 \midrule 	
 	Imposta sui Redditi (\euro)& 31,19\footnote{scaglione del 13,00\: \%} & 94,66\footnote{scaglione del 13,00\: \%} & 1\thinspace 063,88 \footnote{scaglione del 23,00\: \%}\\
	Previdenza Sociale (11,20 \%)(\euro) & 51,48 & 106,16 & 578,35 \\
 \midrule	
	Stipendio Netto (\euro) & 377,00 & 747,08 & 3\thinspace 521,60 \\ 	
 \bottomrule
 \end{tabular} 
\end{table}
\end{savenotes}

Si osserva, in pratica, nel caso del calcolo dell'imposta sui redditi, come, in base allo stipendio imponibile, sia il \textit{Centralinista} sia il \textit{Manager} rientrano nello scaglione del $13 \:\%$, pertanto la loro quota prevista è pari a:
\begin{eqnarray}
\label{eq:imposta_redditi_centralinista}
quota \: centralinista & = & (459,67-219,77) \cdot 0,13 \nonumber \\
					 & = & 31,187 \: \mbox{\euro} \: \simeq 31,19 \: \mbox{\euro} 
\end{eqnarray}

\begin{eqnarray}
\label{eq:imposta_redditi_manager}
quota \: manager & = & (947,90-219,77) \cdot 0,13 \nonumber \\
					 & = & 94,6569 \: \mbox{\euro} \: \simeq 94,66 \: \mbox{\euro}
\end{eqnarray} 

mentre il \textit{CEO} rientra nello scaglione del $23 \:\%$, pertanto la sua quota sarà pari a:
\begin{eqnarray}
\label{eq:imposta_redditi_ceo}
quota \: ceo & = & (5\thinspace 163,83-952,31) \cdot 0,23 + 95,23 \nonumber \\
					 & = & 1\thinspace 063,8796 \: \mbox{\euro} \: \simeq 1\thinspace 063,88 \: \mbox{\euro}
\end{eqnarray} 


\section[Persone Giuridiche]{Persone Giuridiche}
Una persona giuridica, ovvero un ente il cui ordinamento giuridico attribuisce la \textit{capacità giuridica} (diventando, quindi, un \textbf{soggetto di diritto}) è considerata come residente in Albania se ha una struttura permanente, la sede principale, o una sede per la reale gestione degli affari nel Paese.
\subsection{Imposta sul reddito aziendale} 
Tutte le imprese che siano albanesi o straniere registrate ai fini \ac{IVA} sono soggette all'\textit{imposta sul reddito aziendale} calcolata sulla base delle seguenti aliquote:
\begin{itemize}
	\item \textbf{15\%}, per le grandi imprese;
	\item \textbf{imposta semplificata} per le piccole imprese o piccoli imprenditori che realizzano un fatturato annuo lordo inferiore di \textbf{ALL 8 milioni} (circa 58603,77\euro). Le aliquote previste sono:
		\begin{center}
 			\begin{tabular}{SS[table-comparator = true]}
 			\toprule 
 				{Aliquota Applicata (\%)} & {Fatturato Annuale(ALL)} \\
 			\midrule
 				5 & \numrange{5000000}{8000000} \\
 				0 & < 5000000 \\
 			\bottomrule
 			\end{tabular} 
		\end{center}
\end{itemize} 
\subsection{IVA}
E' applicata sulla vendita delle merci e dei servizi a un tasso standard del 20\% e 10\% sulle medicine. La VAT non si applica sulle
esportazioni e sui servizi internazionali come per esempio il trasporto di merci e passeggeri.
\subsection[Apertura Società a responsabilità limitata]{Apertura Società a responsabilità limitata}

\subsection[Costo Complessivo Dipendenti]{Costo Complessivo Dipendenti}
\begin{savenotes}
\begin{table}[htb]
\centering
 \caption{Costo Azienda Dipendenti}
 \begin{tabular}{rD{,}{,}{5.2}D{,}{,}{5.2}D{,}{,}{5.2}D{,}{,}{5.2}}
 \toprule
 	& \multicolumn{1}{c}{Centralinista} & \multicolumn{1}{c}{Manager} & \multicolumn{1}{c}{CEO} & \multicolumn{1}{c}{TOTALE} \\
 \midrule
 	Reddito Imponibile Mensile (\euro)& 459,67 & 947,90 & 5\thinspace 163,83 & \\ 
	Previdenza Sociale (16,70 \%)(\euro) & 76,76 & 158,30 & 862,36 & \\
	\textbf{Costo Mensile Singolo Dipendente} (\euro) & 536,43 & 1\thinspace 106,20 & 6\thinspace 026,19 & \\ 	
	num. dipendenti & 30 & 3 & 1 & 34 \\
	\textbf{Costo Mensile Dipendenti} (\euro) & 16\thinspace 092,90 & 3\thinspace 318,60 & 6\thinspace 026,19 & 25\thinspace 437,69\\
	\textbf{Costo Annuale Dipendenti} (\euro) & 193\thinspace 114,80 & 39\thinspace 823,20 & 72\thinspace 314,28 & 305\thinspace 252,28\\ 	
 \bottomrule
 \end{tabular} 
\end{table}
\end{savenotes}