\chapter[Sistema Fiscale Albanese]{Sistema Fiscale Albanese}
  \label{sec:normative_fiscali_albania}

\section[Persone Fisiche]{Persone Fisiche}
\subsection[Imposta sui Redditi]{Imposta sui Redditi}
Una persona fisica, invece, è soggetta al pagamento delle tasse relative ai guadagni realizzati all'interno del territorio albanese, se non è residente, altrimenti deve pagare le tasse su tutti i guadagni realizzati anche all'estero.
Sono previste le seguenti aliquote:\newline
\begin{savenotes}
\begin{table}[htb]
	\centering
	\begin{tabular}{D{,}{,}{5.2}D{,}{,}{5.2}c}
 \toprule
 	\multicolumn{2}{c}{\textbf{Reddito da lavoro mensile (in ALL)}} & \textbf{Aliquota} \\
 	Da & Fino\ a & \\
 \midrule
 	0 & 30000 & 0\% \\
 	30001 & 130000& 13\% dell'importo superiore ad ALL 30000\\
 	130001 & \ & ALL 13000 + 23\% dell'importo superiore ad ALL 130000 \\
 \bottomrule
 \end{tabular} 
\end{table}
\end{savenotes}

per semplicità riportiamo la precedente tabella con i valori riportati in \textbf{euro}:

\begin{savenotes}
\begin{table}[htb]
	\centering
	\begin{tabular}{D{,}{,}{5.2}D{,}{,}{5.2}c}
 \toprule
 	\multicolumn{2}{c}{\textbf{Reddito da lavoro mensile (in \euro)}} & \textbf{Aliquota} \\
 	Da & Fino\ a & \\
 \midrule
 	0 & 219,77 & 0\% \\
 	219,77 & 952,31 & 13\% dell'importo superiore ad \euro \hspace{0,0150625cm} 219,77\\
 	952,31 & \ & \euro \hspace{0,0150625cm} 95,23 + 23\% dell'importo superiore ad \euro \hspace{0,0150625cm} 952,31 \\
 \bottomrule
 \end{tabular} 
\end{table}
\end{savenotes}
\subsection[Contributi Previdenziali]{Contributi Previdenziali}

\subsection[Calcolo Contributi]{Calcolo Contributi}
Nel nostro caso avremmo la seguente situazione:

\begin{savenotes}
\begin{table}[htb]
\centering
 \caption{Stipendi Dipendenti}
 \begin{tabular}{rD{,}{,}{5.2}D{,}{,}{5.2}D{,}{,}{5.2}}
 \toprule
 	& \multicolumn{1}{c}{Centralinista} & \multicolumn{1}{c}{Manager} & \multicolumn{1}{c}{CEO} \\
 \midrule
 	Reddito Imponibile Mensile (\euro)& 459,67 & 947,90 & 5163,83 \\ 
 	Imposta sui Redditi (\euro)& 31,19\footnote{(aliquota del 13,00 \%) pari a (459,67-219,77)*0,13} & 94,66\footnote{(aliquota del 13,00 \%) pari a (947,90-219,77)*0,13} & 1063,88 \footnote{(aliquota del 23,00 \%) pari a (5163,83-952,31)*0,23+95,23}\\
	Contributi Previdenziali (11,20 \%)(\euro) & 51,48 & 106,16 & 578,35 \\
	Stipendio Netto (\euro) & 377,00 & 747,08 & 3521,60 \\ 	
 \bottomrule
 \end{tabular} 
\end{table}
\end{savenotes}


\section[Persone Giuridiche]{Persone Giuridiche}
Una persona giuridica, ovvero un ente il cui ordinamento giuridico attribuisce la \textit{capacità giuridica} (diventando, quindi, un \textbf{soggetto di diritto}) è considerata come residente in Albania se ha una struttura permanente, la sede principale, o una sede per la reale gestione degli affari nel Paese.
\subsection{Imposta sul reddito aziendale} 
Tutte le imprese che siano albanesi o straniere registrate ai fini \ac{IVA} sono soggette all'\textit{imposta sul reddito aziendale} calcolata sulla base delle seguenti aliquote:
\begin{itemize}
	\item \textbf{15\%}, per le grandi imprese;
	\item \textbf{imposta semplificata} per le piccole imprese o piccoli imprenditori che realizzano un fatturato annuo lordo inferiore di \textbf{ALL 8 milioni} (circa 58603,77\euro). Le aliquote previste sono:
		\begin{center}
 			\begin{tabular}{SS[table-comparator = true]}
 			\toprule 
 				{Aliquota Applicata (\%)} & {Fatturato Annuale(ALL)} \\
 			\midrule
 				5 & \numrange{5000000}{8000000} \\
 				0 & < 5000000 \\
 			\bottomrule
 			\end{tabular} 
		\end{center}
\end{itemize} 
\subsection{IVA}
E' applicata sulla vendita delle merci e dei servizi a un tasso standard del 20\% e 10\% sulle medicine. La VAT non si applica sulle
esportazioni e sui servizi internazionali come per esempio il trasporto di merci e passeggeri.
\subsection[Apertura Società a responsabilità limitata]{Apertura Società a responsabilità limitata}

\subsection[Costo Complessivo Dipendenti]{Costo Complessivo Dipendenti}
\begin{savenotes}
\begin{table}[htb]
\centering
 \caption{Costo Azienda Dipendenti}
 \begin{tabular}{rD{,}{,}{5.2}D{,}{,}{5.2}D{,}{,}{5.2}D{,}{,}{5.2}}
 \toprule
 	& \multicolumn{1}{c}{Centralinista} & \multicolumn{1}{c}{Manager} & \multicolumn{1}{c}{CEO} & \multicolumn{1}{c}{TOTALE} \\
 \midrule
 	Reddito Imponibile Mensile (\euro)& 459,67 & 947,90 & 5163,83 & \\ 
	Contributi Previdenziali (16,70 \%)(\euro) & 76,76 & 158,30 & 862,36 & \\
	\textbf{Costo Mensile Singolo Dipendente} (\euro) & 536,43 & 1106,20 & 6026,19 & \\ 	
	num. dipendenti & 30 & 3 & 1 & 34 \\
	\textbf{Costo Mensile Dipendenti} (\euro) & 16092,90 & 3318,60 & 6026,19 & 25437,69\\
	\textbf{Costo Annuale Dipendenti} (\euro) & 193114,80 & 39823,20 & 72314,28 & 305252,28\\ 	
 \bottomrule
 \end{tabular} 
\end{table}
\end{savenotes}